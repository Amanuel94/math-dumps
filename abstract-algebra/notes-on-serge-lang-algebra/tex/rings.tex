\chapter{Rings}

\begin{proposition}
    For two ideals $\I a, \I b$ of a ring $A$, if $\I a + \I b = A$, then $\I a \cap \I b = \I a \I b.$
\end{proposition}
\begin{proof}
    Clearly, $\I a\I b \subseteq \I a \cap \I b$. Thus, it suffices to prove the contra-positive relation. Since $1 = a + b$ for some $a \in \I a, b \in \I b$,  $c = c\cdot a + c \cdot b$ for all $c \in A$. Of course, if $c \in \I a  \cap \I b$, $c\cdot a + c\cdot b \in \I a \I b$.
\end{proof}

Let $A$ be a ring and let $\lambda: \ZZ \to A$ given by
$$ \lambda(n) = \underset{n \text{ times}}{\underbrace{1_A + \cdots + 1_A}}.$$

Then $\ker \lambda = \gen{n}$ for some $n \geq 0$. If $\gen{n}$ is a prime ideal, then we say $A$ has  characteristic $n$. 

\begin{proposition}
    If $S$ is a set with more than two elements and $A$ is a ring with $1_A \neq 0_A$, then $\Map(S, A)$ is not an integral domain.
\end{proposition}

\begin{proof}
    Let $\set{} \neq T \subset S$
    $$f(x) = \begin{cases}
    1_A \text{ if } x \in  T \\
    0_A \text { if } x \in S - T       
    \end{cases} \text{ and } g(x) = 1_A - f(x).$$
$fg = 0_{\Map(S, A)}.$
\end{proof}

If $\I p$ is a prime ideal in a ring $A$, then it means
\begin{enumerate}
    \item $A/\I p$ is an integral domain.
    \item $xy \in \I p \implies x \in \I p$ or $y \in \I p.$
\end{enumerate}

% \noindent \textit{Every maximal ideal $\I m$  is prime}.
% \begin{proof}
%     $\I m + \gen{x}, x\notin  \I m$ is an ideal containg $\I m$ and thus contains $1$.  
% \end{proof}

\noindent \textit{The ideal $\set{0_A}$ is a prime ideal of $A$ iff $A$ is an integral domain.}
\begin{proof}
    $(\implies )$ $A/\set{0_A} \iso A$, thus $A$ should be an integral domain.

    $(\impliedby)$. If $A$ is an integral domain, then $xy \in  \set{0_A} \implies x = \set{0_A}$ or $y \in \set{0_A}.$
\end{proof}

\begin{theorem}[Chinese Remainder Theorem]
    Let $\I a_1, \dots \I a_n$ be ideals of a ring $A$ such that $\I a_i + \I a_j  = A$ for any $i \neq j$.  Let $x_i$ be elements of $A$. Then there is an element $x \in A$ such that $x \equiv x_i \mod \I a_i.$ 
\end{theorem}

\begin{proof}
    If $n = 2$,  $A = \I a_1 + \I a_2$, and thus $1_A = a_1 + a_2$ for some $a_i \in \I a_i.$ Then $x = x_1 a_1 + x_2a_2$ satisfies the statement.

    If $n > 2$, then $a_i + b_i = 1_A$ for some $a_i \in \I a_1$ and $b_i \in \I a_{j > 1}$. Thus the product $\prod_i (a_i + b_i) = 1_A.$ In other words, 

    $$A = \I a_1 + \prod_{i = 2}^n \I a_i.$$
    By the case for $n=2$,  there is an element $y_1$ such that, 
    $$y_1 \equiv 1_A \mod \I a_1 \text{ and } y_1 \equiv 0_A \mod \big(\prod_{i = 2}^n \I a_i\big)$$
    Since $\prod_{i = 2}^n \I a_i \subseteq \bigcap_{i=2}^n \I a_i$, it follows that $y_1 \in \I a_i$ for all $i > 1$ and therefore, $y \equiv 0_A \mod \I a_i$ for $i > 1$. Carrying out the same procedure in similar fashion to obtain $y_2, \dots, y_n$ such that
    $$y_i \equiv 1_A \mod \I a_i \text{ and } y_i \equiv 0_A \mod \I a_j, j \neq i, $$ 
    we see that $x = \sum_{i = 1}^n x_iy_i$ satisfies the statement of the theorem. 
\end{proof}

\noindent \textit{A non-zero polynomial $f$ of degree $d$ over a commutitive ring $A$  is homogenous iff for every set of $n+1$ algebraically independent elements $u, t_1, \dots, t_n$} over $A$, 
$$f(ut_1, \dots, ut_n) = u^df(t_1, \dots, t_n).$$

\begin{proof}
    Let $f(X) = \sum_{(v)}a_{(v)}X_1^{v_1} \cdots X_n^{v_n}.$ If $f$ is homogenous of degree $d$, $v_1 + \cdots + v_n = d$ for all $a_{(v)} \neq 0$. $f(ut_1, \dots, ut_n) = \sum_{(v)} a_{(v)} (ut_1)^{v_1} \cdots (ut_n)^{v_n}$. Since $A$ is commutitive, this is equal to 
    $\sum_{(v)} a_{(v)} u^{v_1 + \dots + v_n}t_1^{v_1} \cdots t_n^{v_n}$.

    On the other hand, if $f(ut_1, \dots, ut_n) = u^df(t_1, \dots, t_n)$m, then $\sum_{(v)} a_{(v)}u^{v_1 + \cdots + v_n}  = f(u1_A, \dots, u1_A) = u^df(1_A, \dots, 1_A) = u^d\sum_{(v)} a_{(v)}.$ This is a polynomial in $u$ over $A$ and equality is assured iff $u^d = u^{v_1 + \cdots v_n}$. 


\end{proof}

Let $G$ be a \href{https://en.wikipedia.org/wiki/Monoid}{monid} and let $A[G]$ be the set of all mappings $\alpha: G \to A $ such that $\alpha(x) = 0$ for almost all $x \in G$. Addition is defined ordinarily and multiplication is defined as 
$$\alpha \beta (z) = \sum_{xy = z} \alpha(x) \beta(y).$$
Then $A[G]$ is a ring. A more convenient notation can be acheived if we define $a \cdot x$ as 
$$
    a \cdot x(z) = \begin{cases}
        a \text{ if } z = x\\
        0 \text{ if otherwise.}
\end{cases}$$

This way we can define, 
$\alpha = \sum_{x \in G} \alpha(x) \cdot x$, and 
$$\bigg(\sum_{x \in G} a_x \cdot x \bigg) \bigg(\sum_{y \in G} b_y \cdot y \bigg) = \bigg(\sum_{x, y} a_xb_y \cdot xy \bigg)$$

$$\bigg(\sum_{x \in G} a_x \cdot x \bigg) + \bigg(\sum_{x \in G} b_x \cdot y \bigg) = \bigg(\sum_{x\in G} (a_x + b_x) \cdot x \bigg), $$
where $\set{a_z}_{z \in G}$, $\set{b_z}_{z \in G}$ are the elements of $A$, most of them equal to $0$. 

The injective homomorphisms $x \mapsto 1_A \cdot x$ and $a \mapsto a \cdot e$ show that $G$ and $A$ are embedded in $A[G]$. \\ 
% 1.(xy) = (1.x)(1.y    )   


\noindent \textit{Let $A$ be a communtative ring and $S$ be a multiplicative subset \footnote{A subset containting $1_A$ and closed under multiplication}.  For $a, a' \in A$ and $s, s' \in S$, we say}
$$(a, s) \sim (a', s')$$
\textit{if there is $s_1 \in S$ such that}
$$s_1(as' - sa') = 0.$$
\textit{$\sim$ is an equivalance relation.}
\begin{proof}
    Symmetry and Reflexitvness are trivial. Transitivity can be verified as follows. Let $(a, b) \sim (c, d)$ and $(c, d) \sim (e, f)$. Then for some $s_1, s_2 \in S$, we have
    $$s_1ad  = s_1bc$$
    $$s_2de = s_2cf$$
Multiplying both sides of first and the second equation by $s_2f$ and $s_1b$, it follows that $(s_1s_2d)(af - be) = 0.$
\end{proof}
\noindent This construction of ring is called \textbf{ring of fraction of $A$ by $S$, $S^{-1}A$}. The homomorphism $A \mapsto S^{-1}A$ defined by $a \mapsto a/1_A$ is a universal object (See \Ref{universal_object}). If $A$ is an integral domain, then $S^{-1}A$ is the field of fractions.
\\

If $A$ has a unique maximal ideal, it is called \textbf{a local ring.} An intersting example is $A_{\I p} = S^{-1}A$,where $S$ is the multiplicative subset $A - \I p.$

\subsection*{Principal Ideal Domains and Unique Factorization}
Let $A$ be an prinicipal integral domain.
We say $a$ divides $b$ if $b = ac$ for some $c \in A$ 
\begin{definition}
    $d$ is called the greatest common divisor of $a$ and $b$ if and only if $c | a$ and $c | b$ $\implies c | d$.
\end{definition}

\begin{proposition}
    If $d  = \gcd(a, b)$, then $ar + bs = d$ for some $r, s \in A.$ 
\end{proposition}

\begin{proof}
    Let $a = dx$ and $b = dy$. Because $d$ is a gcd of $a$ and $b$, for $c \notin A^*$ $c \mid x \implies c \nmid y$ and vice versa, thus $\gcd(x, y)$ is a unit in $A$. 

    Now, $A \subseteqq \gen{x, y} $. To show that, let $\gen z = \gen{x, y}.$ Since $x, y \in \gen{x, y}$, $x = w_1z$ and $y = w_2z$. But then $z$ should be a unit in $A$ and thus $1_A \in \gen{x, y}.$ The proposition follows directly.
\end{proof}
The proof also shows if
$\gen{a, b} = \gen c$, then $c = \gcd(a, b).$
\begin{definition}
    We call $p \in A$ \textbf{irreducible} if $p = ab$ for some $a, b \in A$, then $\set{a, b} \cap A^* \neq \emptyset$. If $c \in A$ can be written as a product of a unit in $A$ and a product of some irreducibles in $A$, we call the product \textbf{a factorization} of $c$. If every non-zero element of $A$ has a unique factorization (upto commutitivity) we call $A$ a \textbf{unique factorization domain (UFD)} or \textbf{factorial ring}. 
\end{definition}
\begin{theorem}
    If $A$ is a principal ideal domain, then $A$ is a UFD.
\end{theorem}

\begin{proof}
    \noindent \underline{Existence}: Let $S$ be the set of ideals of $A$ generated by elements $a_i$ that don't have factorization. Let $S \neq \emptyset$. Then $\gen{a_1} \in S.$. Consider the chain, 

    \[\gen{a_1} \subsetneqq \gen{a_2} \subsetneqq \cdots \subsetneqq \gen{a_n} \subsetneqq \cdots \]
    Because, $A$ is a principal ideal domain $\cup_i \gen{a_i} = \gen{a}$ for some $a \in A.$ However, $\gen{a_i} \subset \gen{a_{i +1}}$, $a \in \gen{a_n}$ for some $n$ and the chain is finite. Thus if $\gen a \subsetneqq \gen{b}$, then $b$ admits factorization.

    \begin{remark}
        The fact that $A$ is a principal ideal domain is important in constructing the chain. Consider the following chain if  $A = \mathbb{Q}$ , for example
        $$\gen{1/2} \subsetneqq \gen{1/4} \subsetneqq  \cdots  \subsetneqq \gen{1/2^n}  \subsetneqq \cdots $$
        The union of these ideals = $\mathbb{Q}$ which is not a principal ideal.
    \end{remark}

    Now, consider $a$. Clearly, $a$ is not an irreducible. Thus Assume $a = bc$. But $\gen{a} \subsetneqq \gen{b}.$ Thus $b$ (and also $c$) admits factorization and by induction $a$ does making $S$ empty.

    \underline{Uniqueness} First, we prove that irreducibility implies primality. Let $p$ be irreducible and let $p \mid ab$. If $p \nmid a$ then $\gcd(a, p) = 1_A$ and $1_A = ax + py \implies b = abx + pby = p(c'x + by)$ for some $c$.
    
    If $$a = up_1\cdots p_r = vq_1\cdots q_s,$$
    $p_1 \mid q_1 \cdots q_s$ and WLOG, $q_1 = u_1p_1.$ Thus $up_2 \cdots p_r = vu_1q_2\cdots q_s$. The argument completes by induction. 
\end{proof}

% \subsection*{Solutions to Selected Exercises}
% $A$ is commutitive.\\

% \noindent 1. \textit{Suppose $1_A \neq 0_A$ in $A$. Let $S$ be a multiplicative set not containing $0_A$. Let $\I p$ be a maximal element in the set of ideal of $A$ whose intersection with $S$ is  empty. Show that $\I p$ is prime. \\ }
% \hrule
% \vspace*{0.5cm}
% \noindent \textit{ Sol. } Let $xy \in \I p.$ Then $xy \notin S$ which implies $x \notin S$ or $y \notin S$. WLOG, let $x \notin S$. Then 
