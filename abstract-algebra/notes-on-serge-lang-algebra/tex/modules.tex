\chapter{Modules}

The concept of rings is motivated by the properties of a set of \textit{endomorphims} on an (additive) abelian group. Left $R$-modules are the abelian groups $M$ such thaat there is a ring homomorphism $R \to \End(M)$.\\

\noindent \textit{Example:} If $J$ is an ideal of a ring $A$, then we can define an operation of an element $a, b \in A$ on $A/J$ as $a \cdot (x + J) \mapsto ax + J.$ This mapping is an endomorphim of $A/J$ because $a \cdot (x + y + J) = a \cdot (x + J) + a\cdot (y + J)$. We can define the a ring homomorphism from $A \to \End(A/J)$ trivially.  Therefore, $A$ defines a module structure over $A/J$.\\

To show a group $M$ is $A$-module, it suffieces to show that for $a, b \in A$, $x, y \in M$ 
$$a(x + y) = ax + ay \text{ and } (a + b)x = ax + by,$$
These conditions are equivalent to showing there is a ring homomorphism from the actions of elements of $A$ on $M$ to $\End(M)$.\\
\noindent\rule{\textwidth}{1pt}    
    \noindent Some basic constructions from the companion. Let $M$ be an $A$-module.
    \begin{enumerate}
        \item For $N \subseteqq M$, $\set{r \in A : rN = 0}$ forms an left ideal in $A$.
        \item For $N \leqq M$, $\set{r \in A: rM \subseteqq N}$ forms a right ideal of $A$.
        \item For $N \leqq M$, $\set{r \in A: rN \subseteqq N}$ forms a subring.
        \item If $N$ is a submodule, then the ideals in 1 and 2 are $2$-sided. Here, it is important to point out that when $N$ is a submodule, then closure of the actions of $A$ on $N$ is maintained.
    \end{enumerate}
    If $x \in M$, then $Rx \iso R/I$, where $I$ is the annhilator ideal of $\{x\}$ as in 1. \\
    
    \noindent \textit{Every ideal (left, right and 2-sided) and subring of $A$ can be constructed in the above way}\\

    \noindent  
    \noindent\rule{\textwidth}{1pt}

    \begin{definition}
        A \textbf{module-homomorphism} is an additive group homomorphism $f: M \to M'$ from modules $M$ to module $M'$ and such that $f(ax) = af(x).$  
    \end{definition}

    \noindent \textit{If $f$ is module-homomorphism from $M$ to $M'$ then the kernel and the image of $f$ are submodules of $M$ and $M'$ respectively.}
    \begin{proof}
        Clearly, $\ker f \leqq M$ because $f$ is a group homomorphism. Let $a \in A$ and $x \in \ker f$. $f(ax) = af(x) = 0$. Hence, the kernel of $f$ is a submodule of $M$.

        Again, $\Img f \leqq M'$. $af(x) = f(ax) \in \Img f.$

     
    \end{proof}

    \horline\\
    $M'/f(M)$ is a universal(inital) among the modules $N$ with homomorphism $g:M' \to N$ such that $g\circ f = 0.$ That is the following diagram commutes and $\hat g$ is unique:
    \\
    \begin{center}
        
        \begin{tikzcd}[cramped]
            M && {M'} && N \\
            && {} \\
            && {M'/f(M)}
            \arrow["f"{description}, from=1-1, to=1-3]
            \arrow["g"{description}, from=1-3, to=1-5]
            \arrow["c"{description}, from=1-3, to=3-3]
            \arrow["{\hat{g}}"{description}, from=3-3, to=1-5]
        \end{tikzcd}
    \end{center}

    This is dual with the kernel of $f$ which is a terminal object among modules $N$ with homomorphism $g: N \to M$ such that $f \circ g = 0.$ Thus, it is called the \textbf{cokernel} of $f$.    
    \horline

    
    \begin{definition}
        A \textbf{monomorphism} is a module-homomorphism $u: N \to M$ characterized by the exact sequence $0 \to N \overset{u}{\to} M.$ Similarly, an \textbf{epimorphism} is characterized by dual exact sequence $N \overset{u}{\to} M \to 0.$ 
    \end{definition}

    These definitions concide with the definitions of one-to-one homomorphisms and surjective homomorphism in the category of modules over a ring $R$.

    \begin{definition}
        For a commutitive\footnote{the concept of algebras does not make much sense with non-commutative rings} ring $A$, we say $K$ is an $A$-algebra, if $K$ is a module with $E$ a \textbf{\href{https://en.wikipedia.org/wiki/Bilinear_map}{$A$-bilinear map}} $g:E \times E \to E.$ 
    \end{definition}

    \horline

    \noindent In the companion, the following remark is left.\\

    \textit{Let $A$ be a commutative ring. Then 
        $$\text{associative, unital }A\text{-algebra}\ R \equiv \text{Ring $R$ with a homomorphism $f:A \to Z(R).$}$$
    }
    \noindent $f$ is a way of endcoding the bilinear operator, and why it's into the center of $R$ is mainly because we require $a\cdot xy = (a\cdot x)y = x(a \cdot y) :=  f(a)xy = (f(a)x)y = xf(a)y $\\

    Another intersting remark is that algebras are abstractions of the natural structure of $A$-module-endomorphims of a module $M$, $\End_A(M)$, just like rings abstract the endomorphims of an abelian group.  \\
    \horline

    A sequence $\cdots \to A \overset{f}\to B \overset{g}\to C \to \cdots $ is called exact if $\Img f  = \ker g.$ We denote the group of of $A$-homomorphisms from $A$-module $X$ to $Y$ by $\Hom_A(X, Y).$
    \begin{proposition}
        Let $X, X', X''$ and $Y$ be $A$-modules. Then the short sequence
        $$X' \fto{\lambda} X \fto{\mu} X'' \to 0$$
        is exact if and only if 

        $$ \Hom_A(X', Y) \overset{\lambda'}\from \Hom_A(X, Y) \overset{\mu'}\from \Hom_A(X'', Y) \from 0$$
        is exact for all $Y$.
    \end{proposition}
    \begin{remark}
        This proposition is analogous to the duality of linear maps in vector spaces. 
    \end{remark}

    
    \begin{proof}
    Suppose the first sequence is exact. Then the following statements hold:
    \begin{enumerate}[(i)]
        \item $\Img \lambda = \ker \mu$ \label{item:1}
        \item $\Img \mu = X''$.\label{item:2}
    \end{enumerate}
    Let $g \mapsto g \circ \lambda = 0$. Since $\Img \lambda \subseteqq  \ker g,$ $g$ factors through $X/\Img \lambda.$ By \Ref{item:1} and \Ref{item:2}, $X/\Img \lambda \iso \Img \mu =  X''$ which implies $g = f \circ \mu$ for some $f \in \Hom (X'', Y)$. This shows $\ker \lambda' \subseteqq \Img \mu'.$ Similarly, let  $h \circ \mu \in \Img \mu'.$ By \Ref{item:1}, the composition of this with $\lambda$, $h \circ \mu \circ \lambda = 0$, implying $\Img \mu' \subseteqq \ker \lambda'$ (thus $\Img \mu' = \ker \lambda'$).
    The first implication of the proposition follows from the fact that if $f \mapsto f \circ \mu = 0$ for some $f: X'' \to Y$, then $f = 0$  by \Ref{item:2}.
    \\

    The proof of the converse is an easy application of the following common technique: \textbf{To study the consequences of a condition holding for all morphisms of a given sort, consider a universal example.}

    Suppose the second sequence is exact, i.e.,
    \begin{enumerate}[(i)]
        \item $\ker \lambda'  =  \Img \mu'$ \label{item:3}
        \item $\ker \mu' = 0$. \label{item:4}
    \end{enumerate}

    By \Ref{item:3},  $\ker \lambda' \supseteqq  \Img \mu'.$ That is,  for every $Y$ and $f:X'' \to Y$ $f \circ \mu \circ \lambda = 0.$ Now, consider the universal example for all $f$s, i.e.,  the category of morphisms from $X''$ which is $\id$, the identity morphisms. $\id \circ \mu \circ \lambda  = \mu \circ \lambda = 0$ implies $\ker \mu \supseteqq \Img \lambda.$ 

    Similarly, the condition $\ker \lambda' \subseteqq \Img \mu' $ implies for every $Y$, a map $g: X \to Y$ such that $g \circ \lambda  = 0$  can be factored  through $X''.$ The universal object of all morphisms from $X \to Y$ which are $0$ at $\Img \lambda$ is the canonical homomorphism $q: X \to X/\Img \lambda.$ Hence $q = f \circ \mu$ which is obviously $0$ on $\Img \lambda$ and thus $\ker q = \Img \lambda \supseteqq \ker \mu.$
       
    Finally, the universal object of morphisms from $X'' \to Y$  annihilated by $\Img \mu$ is the canonical morphism $p:X'' \to X''/\Img \mu.$ However, \Ref{item:4} implies $p = 0$ and $X'' \iso \Img \mu$ which completes the proof.
    \end{proof}

    Let $\set{M}_{i \in I}$ be a family of submodules of $M$. Then we have the induced homomorphism 
    $$\lambda_*: \bigoplus_{i \in I} M_i \to M$$
    defind by $\lambda_*((x_i)) = \sum  x_i$. If $\lambda_*$ is isomporphism, then we call the family $\set{M}_{i \in I}$, \textbf{direct sum decomposition} of $M$ as we have

    $$\bigoplus M_i = M.$$
    Otherwise, if $\lambda_*$ is only surjective, we can write
    $$M = \sum M_i$$
    \begin{remark}
        This notion is analogous to linear independece and direct sums in linear algebra.
    \end{remark}


    Let $M_1, M_2, N$ be modules. Then we have the following isomorphism of abelian groups

    $$\Hom(M_1 \oplus M_2, N) \iso \Hom(M_1, N) \times \Hom(M_2, N)$$
    $$\Hom(N, M_1 \times M_2) \iso \Hom(N, M_1) \times \Hom(N, M_2)$$

    The first isomorphism follows from the association $f \mapsto (f_1, f_2)$ where $f$ is an element of the LHS group and $f_i: M_i \to N$ are the homomorphisms defined by $f_i = f \circ I_i$. The second one follows with similar associations.


    \begin{proposition}
        Let the following sequence of modules be exact:
        $$0 \to M' \fto{f} M \fto{g} M'' \to 0$$
        The following conditions are equivalent
        \begin{enumerate}
            \item There is a homomorphism $\varphi: M'' \to M$ such that $\id = g \circ \varphi.$
            \item There is a homomorphism $\psi: M \to M'$ such that $\id = \psi \circ f.$
        \end{enumerate}
        If these conditions are satisfied, then we have the following isomporphisms:
        $$M = \ker g \oplus \Img \varphi = \ker \psi \oplus \Img f \iso M' \oplus M''.$$
    \end{proposition}

    The general idea is the exactness of the sequence makes $M$ factorize into $M' \times M/M'$ in group theory terms.
    \begin{proof}
        Let $x \in M$. Then $x - \varphi(g(x)) \in \ker g$ by definition. Thus $x = (x - \varphi(g(x))) + \varphi(g(x)) \in \ker g + \Img \varphi$. This sum is direct because $\ker g \cap \Img \varphi = 0.$ The others isomorphisms follow immediately.
    \end{proof}


    \begin{definition}
        A \textbf{free module} is an $A$-module that admits a basis. 
    \end{definition}

    \begin{proposition}
        Let $M$ be a free module with basis $\set{x_i}_{i \in I}$ and let $\I a$ be a two-sided ideal of $A$. Then 
        \begin{enumerate} 
        \item $\I a M$ is also a submodule of $M$ that is also $\I a$-module.
        \item Each $\I a x_i$ a submodule of $Ax_i$.
        \item We have the module isomorphism 
        $$M/\I a  M \iso  \bigoplus_{i \in I} Ax_i/\I a x_i.$$
        \item $Ax_i/\I ax_i$ is isomorphic to $A/\I a$ as $A$-module
        \item Suppose $A$ is commutative. Then $A/\I a$ is a ring. Furthermore $M/\I aM$ is a free over $A/\I a$ and $Ax_i/\I ax_i$ is a free over $A/\I a$. If $\overline{x_i}$ is the image of $x_i$ under the canonical homomorphism $Ax_i \to Ax_i/\I a x_i$, then $\overline{x_i}$ is the basis of $Ax_i/\I ax_i$.
    \end{enumerate}
    \end{proposition}
    \begin{proof}
        We go through the statements one by one:
        \begin{enumerate}
            \item Let $x \in M$. Then $x = \sum_{i \in I} a_ix_i$ uniquely for $\set{a_i}_{i\in I} \subseteqq A$. By definition, $\I a M = \set{\sum yx : y \in \I a, x  \in M}$. But $yx = \sum_i ya_ix_i = \sum_i y_ix_i \in M$ where $y_i \in \I a$ because $\I a$ is two-sided ideal.
            
            \item Clearly, $\I a x_i \subseteqq Ax_i$. Let $a', b' \in \I a$ and $a, b, c \in A$.  $Ax_i$ is a $A$-module because $(a + b)cx_i = (ac + bc)x_i = acx_i + bcx_i$ and $c(a'x_i + b'x_i) = c(a + b)x_i = (ca + cb)x_i = cax_i + cbx_i.$ The statement follows from $A\I a x_i \subseteqq \I a x_i$
            
            \item By definition, $M = \bigoplus_{i \in I} Ax_i$. Consider the isomprohism 
            $$\sum_{i \in I} a_ix_i \mapsto (a_ix_i)_{i \in I}$$
            which induces the isomorphism
            $$\sum_{i \in I} a_ix_i + \I a M \mapsto  (a_ix_i + \I aM)_{i \in I}.$$
            Since $\I aM$ is a $\I a$-module and  $a_ix_i+ \I  aM = a_ix_i + \I a x_i$, and $Ax_i/\I a x_i$ is an $A/\I a$-module, the statement is true.

            \item Consider the isomoprhism $1_A \mapsto  x_i.$
            \item $A/\I a$ is a ring of cosets of $\I a$.  $M/\I  aM$ is free as the basis $\set{x_i}_{i \in I}$,  serves as a basis for $M/\I aM$ over $A/\I a$.                
        \end{enumerate}
    \end{proof}

     We say an exact sequence of abelian groups
    $$0 \to A \to B \to C \to 0$$
    splits if $B \iso A \oplus C.$

    \noindent \textit{For Example: } The sequence
    $$0 \to \C{2} \fto{x \mapsto x} \C{2} \times \C{2} \fto{-1} \C{2} \to 0 $$
    splits but
    $$0  \to \C{2} \fto{x \mapsto 2x} \C{4} \fto{\mod 2} \C{2} \to 0$$
    does not split.

    \begin{proposition}
        Every surjective module-homomorphism from a $A$-module, $M$ to a \textbf{free} a $A$-module $F$ splits.  
    \end{proposition}
    \begin{proof}
        Let $\phi: M \to F$ be a surjective homomorphism. By the first isomorphism theorem, $F \iso M/\ker \phi$. Let $\set{x_i + \ker \phi}_{i \in I}$ form the basis of $M/\ker \phi$. Define $\psi: M/\ker \phi \to M$ as
        $$\psi\bigg(\sum_{i \in I} a_i x_i + \ker \phi \bigg)= \sum_{i \in I} a_ix_i.$$ 
        Clearly $\phi \circ \psi = \id$
    \end{proof}

    $F$ need not be a free module for $A \to F$ to split. Modules that admit splitting like the above are called \textbf{projective}. Here are four equivalent conditions that are satisfied by a projective module $P$:

    \begin{enumerate}
        \item Given a homomorphism $f: P \to M$ and a subjective homomorphism $g: M' \to M$, there exists a homomorphism $h: P \to M'$ that makes the following diagram commute:\\
        % https://q.uiver.app/#q=WzAsNCxbMSwwLCJQIl0sWzAsMSwiTSciXSxbMSwxLCJNIl0sWzIsMSwiMCJdLFswLDIsImYiXSxbMCwxLCJoIiwyXSxbMSwyLCJnIiwyXSxbMiwzXV0=
       \begin{center}
         \begin{tikzcd}[cramped]
            & P \\
            {M'} & M & 0
            \arrow["h"', from=1-2, to=2-1]
            \arrow["f", from=1-2, to=2-2]
            \arrow["g"', from=2-1, to=2-2]
            \arrow[from=2-2, to=2-3]
        \end{tikzcd}
       \end{center}
        \item The exact sequence $0 \to M' \to M \to P \to 0 $ splits
        \item There exists a module $M$ such that $P \oplus M$ is free.
        \item The functor $M \mapsto \Hom_A(P, M)$ is exact.
    \end{enumerate}

    \begin{proof}
        We only leave the proof of $(4) \implies (1)$ as the rest is found in the book. Consider (4) is true, i.e, if $0 \to M'' \to M' \fto{g} M \to 0$ is exact, $0 \to \Hom_A(P, M'') \to \Hom_A(P, M') \fto{\lambda} \Hom_A(P, M) \to 0$ is also exact. Since $\lambda$ is surjective,  for any $f \in \Hom_A(P, M)$, we can find $h \in \Hom_A(P, M')$ such that $\lambda(h) = g \circ h = f$.  
    \end{proof}

    \begin{proposition}
        Let $V$ be a vector space. Let $\Gamma$ be the set of generators of $V$ and $S$ be a set of any linearly independent elements. Then, there is a basis $\m B$ such that $S \subseteqq \m B \subseteqq \Gamma$.  
    \end{proposition}

    \begin{proof}
        Let $\m I$ be the sets $T \supseteqq S$ that are linearly independent. Assuming $V \neq \set{0}$, $\m I$ is non-empty. Clearly $\m I$ is a poset by ascending inclusion. Since if $T_i \subseteqq T_{i+1} \in \m I$ then $T_i \cup T_{i+1}$ is linearly independent making $\m I$ an inductively orderd set. By zorns lemma, there is a maximal element of $\m I$. Let's call that $\m B$ and let $\gen{B} = W$. If $W \neq V$, then there is $x \in V$ such that $x \neq \sum_{y \in \m B} a_y y$ making $\m B \cup x$ linearly independent and contradictng maximality of $\m B.$ Thus $V = W.$    
    \end{proof}


    \begin{proposition}
        Let $V, U$ be  vector spaces over field $K$ and let $V \fto{f} U $ be homomorphism. Then we have
        $$\dim_K V = \dim_K \ker f + \dim_K \Img f.$$ 
    \end{proposition}

    \begin{proof}
        Let $\set{w_i}_{i \in I}$ and $\set{u_i}_{i \in I}$ be the basis of $\ker f$ and $\Img f$ resp. Let $\set{v_i}_{i \in I}$ be a family of elements such that $f(v_i) = u_i.$  Let $x \in V$. Then we have,
        $$f(x) = \sum_{i \in I} a_i u_i,$$
        where $\set{a_i}_{i \in I}$ is a family in $K$ such that all except finit of them are $0$. This implies, 
        $$y = x - \sum_{i \in I} a_iv_i \in \ker f.$$
        However, $\ker f$ is a vector field and $y = \sum_i b_iw_i.$ This implies 
        $$x = \sum_{i \in I} a_i v_i + \sum_{j \in I} b_j w_j.$$
        Proving $\set{v_i, w_i}_{i \in I }$ generates $V$. It remains to show that this generator is linearly indepenendent.

        Let $0 = \sum_{i \in I} a_i v_i + \sum_{j \in I} b_j w_j.$ Then $f(\sum_{i \in I} a_i v_i + \sum_{j \in I} b_j w_j) = 0 + \sum_{i \in I} a_i f(v_i) + 0 = \sum_{i \in I} a_i u_i = 0 \implies a_i = 0 \implies b_j = 0.$
    \end{proof}

    \horline  

    \noindent An important insight from the companion:
    \textit{A free left $R$-module with rank $n$ is isomorphic to a standard\footnote{By standard, we mean where the action of $R$ is trivial as in linear algebra} module $R^n.$} This helps us derive the following facts about modules over non-field ring:

    \begin{itemize}
        \item \textit{If $R \fto{f} S$ is a homomorphism and $m, n$ are positive integers such that $R^m \iso R^n$, then $S^m \iso S^n$.}\\
        If $\mathcal{M}$ is a (isomorphic) transformation from $R^m \to R^n$, then is $f(\mathcal{M})$ is too from $S^m \to S^n$.

        \item \textit{If there is a homomorphism onto a  field (division ring), then all left $R$-modules have a fixed number of elements in their basis.}
    
        
        This follows by taking $f = R \mapsto R/I$ where $I$ is a maximal ideal.
        
        
    \end{itemize}

    \textbf{Warning:} Modules over non-commutative  rings do not  necessarily have unique ranks. 

    \subsection*{Dual Space and Dual Module}
    Let $E$ be a free module over a commutitive ring $A$. We denote the \textbf{dual module}, $\Hom_A(E, A)$, of $E$ by $\dual{E}$ and we call the elements of $\dual{E}$ as \textbf{functionals.}

    If $x \in E$, then $x$ induces a map $\gen{x, -}$ from $\dual{E}$ to itself defined by $\gen{x, f} = f(x).$ 

    \horline
    The map $\theta: E \to \ddual{E}$ is not surjective for the following reason. In infinite-dimensional modules over a field $A$, $\ddual{E}$ is also infinite dimensional. However, $x$ can be expressed as a linear combination of the basis of $E$ and so is $\theta(x).$\\
    \horline

    \begin{proposition}
        If $E$ is free, so is $\dual{E}.$ Moreover, $\rank E = \rank \dual{E}$
    \end{proposition}

    \begin{theorem}
        Let $E$ be finite dimensional. The map $x \fto{\phi} (f \mapsto \gen{x, f})$ is an isomorphism from $E$ to $\ddual{E}.$
    \end{theorem}

    \begin{theorem}

        Let $U, V, W$ be finite-dimensional free modules over commutative ring $A$. If the sequence
        $$0 \to W \to V \to U \to 0$$
        is exact, then so is
        $$ 0 \to \dual{U} \to \dual{V} \to \dual{W}.$$
    \end{theorem}

    \horline
    \noindent \textit{Why it is called a sequence splits?}
    A short sequence
    $$0 \to C \to C \oplus B \fto{g} B \to  A$$
    is \textbf{splits} into

    $$0 \to C \to C \to 0 \to 0$$
    $$0 \to 0 \to B \to B \to 0$$

    We require a right inverse map $g'$,i.e., that satisfies $\id = g \circ g': B \to C \oplus B$, to say so, the action of this map on $C$ would be $0$ and the action on $B$ would be $g^{-1}$ \\
    \horline

    \subsection*{Modules over Principal Ideal Domains}
    \begin{theorem}
        Let $R$ be a principal ideal domain and let $F$ be a free $R$-module. If $M$ is a submodule of $F$, then $M$ is free with rank less than or equal to $\rank F$. 
    \end{theorem}
    \begin{proof}[sketch]
        Let $M_i$ be the submodule of $M$ generated by the basis subset $\{x_1, \dots ,x_i\}$. Let $\I a_{i+1}$ denote the set of coffecients of $x_{i+1}$ in $M - M_i.$ If $\I a_{i + 1} = 0$, we are done. If not, observe that $RM_i \subseteqq M_i$ and $\I a_{i+1} = \gen{a_{i+1}}$ for some $a_{i+1} \in  R.$ Let $w := \sum_{j \leqq i} b_jx_j + a_{i+1}x_{i+1}$. Then $M_{i+1} = M_i + Rw.$
    \end{proof}
    \begin{remark}
        The PID nature of $R$ permits the constructions of  \textit{generators} $w_i$ of $M$ corresponsing   to the generators $x_i$
    \end{remark}

    \noindent \textbf{NB:} Finitely generated modules are factor modules of a free module.

    \begin{definition}
        An $R$-module $M$ is called a \textbf{torsion} module if for some $x \in M$, there is an element $a \in R$ such that $ax = 0$. We denote the module that contain all torsion elements by $\tor M.$
    \end{definition}

    \begin{theorem}
        Let $E$ be finitely generated. The factor module $E/\tor E$ is free and there is a free submodule $F$ of $E$ such that 
        $$E = \tor E \oplus F.$$
    \end{theorem}   

    Modules ove PID exhibit similar characterstics as abelian groups. For example, the cyclic $p$-groups are analogous to a moule generated by an element $x$ modulo a prime ideal, i.e $Rx/(p)x$. We call a module of type $(r_1, \dots, r_k)$ if is a product of modules isomorphic to $R/(p^{r_i}).$ The following two theorems support the similarity even more by stating the equivalent statements to the fundamental theorem of abelian groups.
    
    \begin{theorem}
        Let $R$ be a princial ideal domain and let $E$ be a finitely generated torsion 
        module over $R$. Let $E(p)$ denote all elements of $E$ with exponent\footnote{An exponent of a module $M$ (an element of a module $x$ resp.) is an element $m$ of $R$ such that $mE$ (resp. $mx$) is 0.} that is a power of a prime element $p \in R$. Then $E$ has the decomposition

        $$E = \bigoplus_p E(p), $$
        where the direct sum is over $p$ such that $E(p) \neq 0.$ Moreover, for each $p$, we have
        
        $$E(p) = R/(p^{v_1}) \oplus \cdots \oplus R/(p^{v_r})$$
        with $1 \leqq v_1  \leqq \cdots \leqq v_r$ that are determined uniquely.
    \end{theorem}
    \noindent $E_m := $ the kernel of the map $x \mapsto mx$ in $E$.
    \begin{proof}
        Let $a$ be an exponent of $E$. Consider the map $x \mapsto ax.$ Let $a = bc$ with $(b,c) = (1)$. Let $xb + yc = 1$. Then $v = vxb + vyc$ where $vxb \in E_c$ and $vyc \in E_b$. Moreover, $E_b \cap E_c = 0.$ Thus $E_a = E_b \oplus E_c.$ By induction, the stated decomposition of $E$ follows. 

        Next, we show that $E(p)$ is a direct sum as stated. 
        \\\horline The intuition for such decomposition of $E(p)$ comes from boxing all elements of $E(p)$ with the same period\footnote{A period $T$ of an element $x$ is an element of $R$ such that the kernel of the mao $a \mapsto ax $ equals $\gen{T}$} into a direct summand.\\
        \horline
        We will use induction. Consider the canonical map from $E(p) \to E(p)/(x)$ where $x$ is an element of $E(p)$ with maximal period, $p^r$. Suppose $\set{\overline{y_1}, \dots, \overline{y_m}}$ are independent\footnote{We call a family of elements $\set{y_i}$ of a module $M$ independent if $\sum_i a_iy_i = 0 \iff a_iy_i = 0\ \forall i$} elements of $E(p)/(x)$ with representatives $\set{y_1,  \dots, y_m}$ in $E(p)$. If $p^{n_i}$ is the period of $\overline{y_i}$, then $p^{n_i}y_i = p^scx$ for some $c \in R$, $p \nmid c.$ By assumption, $r \geqq s$, thus $p^{n_i - s  + r} = 0 \implies n_i - s + r \leqq r \implies n_i \leqq s.$ Therefore the element $y_i - p^{s - n}cx$ is well-defined and has period equal to that of $\overline{y_i}.$

        Moreover the set $\set{x, y_1, \dots, y_m}$ is independent because if $bx + \sum_ia_iy_i  = 0$, then $\sum_i a_i\overline{y_i} = 0$ which can not happen unless $a_i\overline{y_i} = 0$ for all $i$. But by previous part of the proof, this implies all period $c_i \mid a_i \implies a_iy_i = 0$ and $bx = 0.$  
        
        Thus, $E(p)$ has $m+1$ independent elements $x, y_1 \dots, y_m$. It is clear that $(x, y_1, \dots, y_m) = (x)\oplus(y_1)\oplus\cdots\oplus(y_m)$ by independece. Note that if $w \in E$ has period $t$, then $(w) \iso R/\gen{t}.$ This proves the existence of such decomposition.

        Uniquness of the decomposition follows as following. Let $(s_1, \dots, s_m)$ and $(r_1, \dots, r_n)$ be two types of $E(p)$ with $s_i \leqq s_{i+1}$ and $r_i \leqq r_{i+1}$. WLOG, let $s_i < r_i$ be the first different entries. Clearly, there is an element $x \in E(p)$ with period $p^{s_i}$. However, no such element exist in $R/(p^{r_{i}}) \oplus \cdots \oplus R/(p^{r_n}).$ Thus $s_i = r_i$.   
    \end{proof}


    \begin{remark}
        The proof of theorem 7.8 on the book utilizes a trick to select a basis set with particular property. The trick relies (generally speaking) on the fact that functionals capture the properties of basis.

        \noindent \textit{For example}: The dimenstion of a free module $M$ is equal to $\max_{\lambda \in \dual{M}} \dim \lambda(M).$
    \end{remark}


      \subsection*{Direct and Inverse Limits}
    Let $I$ be a \href{https://en.wikipedia.org/wiki/Directed_set}{directed set}. 
    Let $\set{A_i}_{i \in I}$ be a family of $A$-modules and let$\set{f_{i,j}:A_i \to A_j}$ be a family of $A$-homomorphism satisfying
    $$
        f_{i, i} = \id$$
        $$f_{i, k} = f_{j, k} \circ f_{i, j} \text{ if } i < j. 
    $$
    We call this family of morphisms, \textbf{a directed family of morphisms}. When we have a family like $\set{A_i}$, we want to study their properties together. The \textbf{direct limit} has the required algebraic properties to do so and it's defined as follows.

    Consturct a category $\mathcal{C}$ by defining $\Ob(\mathcal{C})$ as the pair $(A, f_i)$ with $A$ in the family of modules and $f_i: A_i \to A$ that makes the following diagram commute

    \begin{center}
    \begin{tikzcd}
    A_i \arrow[dr, "f_{i,j}"'] \arrow[rr, "f_i"] & & A \\
    & A_j \arrow[ur, "f_j"'] &
    \end{tikzcd}
    \end{center}

    where the morphisms are $f_i$ themselves. The direct limit $(B = \dirlim A_i, h_i)$ is the universal object of this category, i.e., for every $(C, g_i)$ in this cateory there is a unique homomorphism $t$ that makes the following diagram commute


    % \noindent \textit{There is a unique homomorphism from the basis of a module $M$ to any subset of another module $N$.}
    % ker + F
    % A/ker f is free  x ... x + ker f, A = u + x ... x  ker + A
    % a = ker f + \sum xi
    % Z/4 , Z/4/Z/2 Z/2
    % 
% \noindent The reason why $f$ is into the center of $R$ in particular is closly tied to the identity $a \cdot x = (a \cdot 1_R)x$.

% A/a xi iso Axi / axi....   Axi -> (A/a)xi  f(aixi) = (ai mod a) xi   
    % ra in I
    % M = A
    % N = {r in A such that gr = 0 for all g in S} if I = <S>
    % N = <S> rm
    % r.m = sum gi.ai 
    % N  = {r in A such that }
    
    
