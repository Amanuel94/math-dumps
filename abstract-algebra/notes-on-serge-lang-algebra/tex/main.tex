\documentclass{book}

\usepackage{amsmath}
\usepackage{amssymb}
\usepackage{amsthm}
\usepackage{hyperref}
\usepackage{bookmark}
% \usepackage{fbb}
% \usepackage{newpx}
% \usepackage{libertine}
% \usepackage{crimson}
% \usepackage{bm}
\usepackage{cochineal}

% \usepackage{eulervm} % math
\usepackage{mathpazo} % math

\usepackage{tikz-cd}
\usepackage{enumerate}

\newtheorem{theorem}{Theorem}
\newtheorem{lemma}[theorem]{Lemma}
\newtheorem{corollary}[theorem]{Corollary}
\newtheorem{conjecture}[theorem]{Conjecture}
\newtheorem{proposition}[theorem]{Proposition}
\theoremstyle{definition}
\newtheorem{definition}[theorem]{Definition}
\newtheorem{example}[theorem]{Example}
\theoremstyle{remark}
\newtheorem{remark}[theorem]{Remark}

\hypersetup{
    colorlinks=true,
    linkcolor=blue,
    urlcolor=cyan,
    citecolor=red
}

\begin{document}

\title{Notes on Serge Lang's Algebra}
\author{Amanuel Tewodros}
\maketitle

\tableofcontents

% commands

\newcommand{\size}[1]{|#1|}
\newcommand{\idx}[2]{[#1 : #2]}
\newcommand{\gen}[1]{\langle #1 \rangle}
\newcommand{\set}[1]{\{#1\}}
\newcommand{\normal}{\trianglelefteq}
\newcommand{\subgroup}{\leq}
\newcommand{\normalizer}[1]{\text{N}_{#1}}
\newcommand{\centerizer}{Z}
\newcommand{\aut}{\text{Aut}}
\newcommand{\inn}{\text{Inn}}
\newcommand{\syl}{\text{Syl}}
\newcommand{\sym}{\text{Sym}}
\newcommand{\alt}{\text{Alt}}
\newcommand{\id}{\text{id}}
\newcommand{\im}{\text{im}}
\newcommand{\orbit}[1]{\mathcal{O}_{#1}}
\newcommand{\ZZ}{\mathbb{Z}}
\newcommand{\C}[1]{\mathbb{Z}/#1\mathbb{Z}}
\newcommand{\iso}{\cong}
\newcommand{\dihedral}[1]{D_{#1}}
\newcommand{\dicyc}{\text{DiC}}
\newcommand{\hol}{\text{Hol}}
\newcommand{\I}[1]{\mathfrak{#1}} 
\newcommand{\Map}{\mathrm{Map}}
\newcommand{\nequiv}{\not \equiv}
\newcommand{\cat}[1]{\mathcal{#1}}
\newcommand{\Cat}{\mathcal{C}}
\newcommand{\Mor}{\text{Mor}}
\newcommand{\Ob}{\text{Ob}}
\newcommand{\End}{\text{End}}
\newcommand{\horline}{\noindent\rule{\textwidth}{1pt} \newline}
\newcommand{\Img}{\mathrm{Im}\ }
\newcommand{\Hom}{\mathrm{Hom}}
\newcommand{\fto}[1]{\overset{#1}{\to}}
\newcommand{\from}{\leftarrow}
\newcommand{\m}[1]{\mathfrak{#1}}
\newcommand{\dual}[1]{#1^{\vee}}
\newcommand{\ddual}[1]{#1^{\vee \vee}}
\newcommand{\rank}{\mathrm{rank}}
\newcommand{\tor}[1]{#1_{\mathrm{tor}}}
\newcommand{\dirlim}{\underrightarrow{\lim}}
\newcommand{\invlim}{\underleftarrow{\lim}}
% \newcommand{\rank}[1]{def}
\chapter{Groups}

\begin{theorem}[Sylow Theorems]
    Let $G$ be a finite group with $p$ divides $\size{G}$, where $p$ is a prime. Then
    \begin{enumerate}
        \item There exists a Sylow $p$-subgroup of $G$.
        \item The number of Sylow $p$-subgroups of $G$ is congruent to $1$ modulo $p$ and divides $\size{G}$.
        \item All Sylow $p$-subgroups of $G$ are conjugate.
    \end{enumerate}
\end{theorem}

\begin{proof}
    If $H \subgroup G$ with $\idx{G}{H}$ coprime with $p$, then by induction $H$ and therefore $G$ contains a Sylow $p$-group. Otherwise, by the class equation,
    $$\size{G} = \size{\centerizer(G)} +  \sum_{x} \idx{G}{N_x(G)}, $$
    it follows $\centerizer(G)$ is divisible by $p$ and thus $\gen{g} \subgroup \centerizer(G)$ for some $g \in \centerizer(G)$ with exponenet = $p$. Inducting on the order of $G$, $G/\gen{g}$ contains a Sylow $p$-subgroup, say $S/\gen{g}$ that is the image of $S \subgroup G$ that is a Sylow $p$-subgroup of $G$. 
    
    Let $P, Q \in \syl_p(G)$. $P$ does not normalize $Q$ because otherwise $PQ \subgroup G$ and $p^m = \size{PQ} > \size{P}$, a contradiction. Let $S = \set{P_1, \dots, P_k}$ be the conjugates of $P$ and let $\orbit{i}$ be the orbit of $P_i$ by the action $P$ on the  set $S$ by conjugation. Then $\size{\orbit{i}} = \idx{P}{N_P(P_i)} = \idx{P}{N_G(P_i) \cap P} = \idx{P}{P_i \cap P} \implies k = 1 \mod p.$

    If $P, Q \in \syl_p(G)$ are not conjugates, then $Q$ is not conjugate with conjugates of $P$.
    Consider the action of the elements of $Q$ on the set $\set{gPg^{-1} : g \in G} = \set{P_1, \dots, P_m}$. Then 
    $$\size{\orbit{P_i}} = \idx{Q}{N_Q(P_i)} = \idx{Q}{P_i \cap Q}, $$
    where the latter equality follows because $P_i(N_G(P_i) \cap Q)$ is a $p$-group that contains $P_i$ with order $\leq \size{P_i}$(a Sylow $p$-group) and thus $ N_G(P_i)\cap Q \leq P_i$. Since $Q$ is not a conjugate of $P$, $\idx{Q}{Q \cap P_i} = p^k, k > 0$ and $ \orbit{P_i}$ is divisible by $p$ and the number of conjugates of $P$ which is $\sum_i \size{\orbit{P_i}} = 0 \mod p$, a contradiction.
    \end{proof}


    \begin{theorem}
        If $\size{G}= pq$ for primes $p < q$, then $G = \ZZ/pq\ZZ$ if $p \nmid q - 1$ else $G = \C{pq}$ of $G =  \ZZ/q\ZZ \rtimes \C{p} $ for some non-trvial semi-direct product. 
    \end{theorem}

    \begin{proof}
        If $q > p$, $n_q = 1$ and thus $Q \in \syl_q(G)$ is normal. $|\aut(\C{q})| = q - 1$, therefore, there is a nontrivial map $\phi: \C{p}\to \aut(\C{q})$ if $p \mid q - 1$  
    \end{proof}
    \begin{theorem}[Fundamental Theorem of Finitely Generated Abelian Groups]

        Let $A$ be a finite abelian group and let $A(p)$ be the subgroup of all elements with order that is a power of $p$. Then 
        $$\prod_{A(p) \neq \set{1}} A(p) = A.$$
    
    \end{theorem}

    \begin{proof}
        Clearly the map $\phi: \prod_p A(p) \to A$ defined by $\phi((x_p)) = \prod_p x_p $ is an endomorphism.  We show that $\phi$ is injective and surjective. Let $\phi((x_p)) = 1$ for some $x = (x_p) \in \prod_p A(p)$. Let $q$ be a prime with $A(q) \neq \{1\}$. Then 
        $$x_q = \prod_{ p \neq q} x_p^{-1}.$$
        Let $m$ be the least common multiple of the primes powers  on the right hand side, i.e. powers of $p \neq q$. Then $x^{m}_q = 1$. But, $x_q^{q^r} = 1$ too. Consequently, $x_q^{(m, q^r)} = x_q^{1} = x_q = 1$. Thus $\prod_p x_p = 1$ iff all $x_p = 1$ and $\ker \phi = \set{1}.$

        To prove surjectivity, let $x \in A$ with $x^m = 1$ such that $m = \prod p_i^{r_i}$. By Euclidean algorithm,  $1= \sum_i u_i\prod_{j\neq i} p_j^{r_j}$ and thus $x = \prod_i x^{u_i\prod_{j\neq i} p_j^{r_j}}$ with 
        $ x^{u_i \prod_{j\neq i} p_j^{r_j}} \in A(p_i).$
    \end{proof}
    
   
    \textbf{Why nilpotence and the existence of normal Sylow sub-groups are equivalent?}: If $P, Q \in \syl_p(G)$ then $N_P(Q) = P\cap Q < P, Q$ and thus $Z(G)$ is always $ < G$. Thus $P = Q \iff G$ nilpotent.  

    \textbf{The number of ways $G$ acts on $H$}: $= \#$ of homomorphisms from $G$ to $\aut(H) = \#$ subgroups of order $|G|/|H^*|.$ 

    \begin{theorem}
        If $n \geq 5$ then $S_n$ is not solvable.
    \end{theorem}

    \begin{proof}
        Let $S_n$ decompose as $S_n = H_m \supset \cdots \supset H_0 = \set{1}$. Clearly, $S_n$ contains all $3$-cycles. We also know since $H_n/H_{n-1}$ is abelian $(abc)(ade)(acb)(aed) = (adebc)(aedcb) = (abd) \in H_{m-1}$. By induction all 3 cycles are in $\{1\}$, a contradiction.
    \end{proof}

    \begin{theorem}
        $A_n$ is simple for all $n \geq 5$.
    \end{theorem}
    \noindent \textit{A priori}: $A_n$ can be generated by $3$-cycles and $3$-cycles are conjugates.
    \begin{proof}
        Let $N \normal A_n.$ Let $\sigma \in N$. We show that $\sigma$ is a $3$-cycle or $\sigma = \id.$ The former implies $N = A_n$ and the latter implies $N$ is the trivial subgroup.  Let $\sigma$ have the maximal number of fixed points in $N$.

      Lrt all $\sigma$'s orbits have size $2$ and it does not fix elements $i, j$. If $\sigma$ is $(ijk)$ for some $k$, we are done. Otherwise,  $\gen{\sigma} > \gen{(ij)(rs)}$ for some $r, s$ because $\sigma$ is an even permutation and not a $3$-cycle. Let $\tau = (rsk)$ for some $k$. Then $\tau' = \tau \sigma \tau^{-1} \sigma^{-1}  \in N$. But $\tau' = (i, j)\sigma$ contradicting $\sigma$ fixes the maximal number of points. Thus at least one $\sigma$'s orbit has more than $2$ elements.

    Therefore, $\sigma = (ijk)(rs) \theta$ where $\theta$ is possible identity permutation. By similar argumenta as above picking $\tau' = (rsk)$, $\sigma$ can not be the element of $N$ with maximal fixed points unless it contains all of $A_n$.  
    \end{proof} 

    \subsection*{Properties of Common Non-Abelian Groups}
    \begin{itemize}
        \item \textit{Dihedral Group}: $\dihedral{2n}$
        \begin{itemize}
            \item  $\iso \C{n} \rtimes \C2$ acting by inversion
            \item $=\set{a, b | a^n, b^2, baba}$
        \end{itemize}

        \item \textit{Binary Dihedral Group/ Dicyclic Group}: $\dicyc(4n)$
        \begin{itemize}
            \item  $\iso \C{n} \rtimes \C4$ acting by inversion
            \item $=\set{a, b | a^n, b^4, baba}$
        \end{itemize}

        \item \textit{Generalized Quaternions}: $Q_{2^{n+2}}$
        \begin{itemize}
            \item  $\iso \C{2^n} \rtimes \C4$ acting by inversion
            \item $=\set{a, b | a^{2^n}, b^4, baba}$
        \end{itemize}

        \item \textit{Holomorph Group}: $\hol(G)$
        \begin{itemize}
            \item  $\iso G \rtimes \aut(G)$ 
            \item if $G$ is $\C{p}$, $p$ prime, $\hol(G)$ is isomorphic to the \textit{generalized affine group} 
        \end{itemize}
    \end{itemize}

    \subsection*{Notes on Category Theory}
    \begin{itemize}
        \item A category $\cat{C}$ is a collection of \textbf{objects} $\Ob(\cat{C})$, along with a set of maps, called \textbf{morphisms} between any two objects $A,B \in \Ob(\Cat)$ denoted by $\Mor(A, B)$.
        \item Morphisms follow the law of composition.
        \item Three axioms
        \begin{enumerate}
            \item \textbf{CAT 1} $\Mor(A, B)$ and $\Mor(A', B')$ are disjoint unless $(A, B) = (A', B')$, in which case they are equal.
            \item \textbf{CAT 2} For every $A \in \Ob(\Cat)$, there exists a morphism, $\id_A$ in $\Mor(A, A)$ that acts as a left and right identity for the elements of $\Mor(A, B)$ and $\Mor(B, A)$ resp. for all $B$.
            \item \textbf{CAT 3} The law of composition of morphisms is associative.
        \end{enumerate}
        \item The \textbf{operation} of a group $G$ on an object $A \in \Ob(\Cat)$ is a homomorphism from $G$ to $\aut(A).$ It is also called a \textbf{representation.}
        \item Given a category $\Cat$, we can construct a new category $\cat{D}$ where the objects are the morphisms of $\Cat$ and the morphisms between two objects $f, f'$ are  defined by a pair of momorphism $(\phi, \psi)$  that make the following diagram commute:
 
\[\begin{tikzcd}
	A && B \\
	{A'} && {B'}
	\arrow["f", from=1-1, to=1-3]
	\arrow["\phi"', from=1-1, to=2-1]
	\arrow["\psi", from=1-3, to=2-3]
	\arrow["{f'}"', from=2-1, to=2-3]
\end{tikzcd}\]

\label{universal_object}\item An object $P$ of a category $\Cat$ is called \textbf{universally attracting} (resp. \textbf{universally repelling}) if there is exists a \textit{unique} morphism from (resp. to) every object to(resp. from) $P$. If it is both, it is called \textbf{universal object}.

\end{itemize}    
\chapter{Rings}

\begin{proposition}
    For two ideals $\I a, \I b$ of a ring $A$, if $\I a + \I b = A$, then $\I a \cap \I b = \I a \I b.$
\end{proposition}
\begin{proof}
    Clearly, $\I a\I b \subseteq \I a \cap \I b$. Thus, it suffices to prove the contra-positive relation. Since $1 = a + b$ for some $a \in \I a, b \in \I b$,  $c = c\cdot a + c \cdot b$ for all $c \in A$. Of course, if $c \in \I a  \cap \I b$, $c\cdot a + c\cdot b \in \I a \I b$.
\end{proof}

Let $A$ be a ring and let $\lambda: \ZZ \to A$ given by
$$ \lambda(n) = \underset{n \text{ times}}{\underbrace{1_A + \cdots + 1_A}}.$$

Then $\ker \lambda = \gen{n}$ for some $n \geq 0$. If $\gen{n}$ is a prime ideal, then we say $A$ has  characteristic $n$. 

\begin{proposition}
    If $S$ is a set with more than two elements and $A$ is a ring with $1_A \neq 0_A$, then $\Map(S, A)$ is not an integral domain.
\end{proposition}

\begin{proof}
    Let $\set{} \neq T \subset S$
    $$f(x) = \begin{cases}
    1_A \text{ if } x \in  T \\
    0_A \text { if } x \in S - T       
    \end{cases} \text{ and } g(x) = 1_A - f(x).$$
$fg = 0_{\Map(S, A)}.$
\end{proof}

If $\I p$ is a prime ideal in a ring $A$, then it means
\begin{enumerate}
    \item $A/\I p$ is an integral domain.
    \item $xy \in \I p \implies x \in \I p$ or $y \in \I p.$
\end{enumerate}

% \noindent \textit{Every maximal ideal $\I m$  is prime}.
% \begin{proof}
%     $\I m + \gen{x}, x\notin  \I m$ is an ideal containg $\I m$ and thus contains $1$.  
% \end{proof}

\noindent \textit{The ideal $\set{0_A}$ is a prime ideal of $A$ iff $A$ is an integral domain.}
\begin{proof}
    $(\implies )$ $A/\set{0_A} \iso A$, thus $A$ should be an integral domain.

    $(\impliedby)$. If $A$ is an integral domain, then $xy \in  \set{0_A} \implies x = \set{0_A}$ or $y \in \set{0_A}.$
\end{proof}

\begin{theorem}[Chinese Remainder Theorem]
    Let $\I a_1, \dots \I a_n$ be ideals of a ring $A$ such that $\I a_i + \I a_j  = A$ for any $i \neq j$.  Let $x_i$ be elements of $A$. Then there is an element $x \in A$ such that $x \equiv x_i \mod \I a_i.$ 
\end{theorem}

\begin{proof}
    If $n = 2$,  $A = \I a_1 + \I a_2$, and thus $1_A = a_1 + a_2$ for some $a_i \in \I a_i.$ Then $x = x_1 a_1 + x_2a_2$ satisfies the statement.

    If $n > 2$, then $a_i + b_i = 1_A$ for some $a_i \in \I a_1$ and $b_i \in \I a_{j > 1}$. Thus the product $\prod_i (a_i + b_i) = 1_A.$ In other words, 

    $$A = \I a_1 + \prod_{i = 2}^n \I a_i.$$
    By the case for $n=2$,  there is an element $y_1$ such that, 
    $$y_1 \equiv 1_A \mod \I a_1 \text{ and } y_1 \equiv 0_A \mod \big(\prod_{i = 2}^n \I a_i\big)$$
    Since $\prod_{i = 2}^n \I a_i \subseteq \bigcap_{i=2}^n \I a_i$, it follows that $y_1 \in \I a_i$ for all $i > 1$ and therefore, $y \equiv 0_A \mod \I a_i$ for $i > 1$. Carrying out the same procedure in similar fashion to obtain $y_2, \dots, y_n$ such that
    $$y_i \equiv 1_A \mod \I a_i \text{ and } y_i \equiv 0_A \mod \I a_j, j \neq i, $$ 
    we see that $x = \sum_{i = 1}^n x_iy_i$ satisfies the statement of the theorem. 
\end{proof}

\noindent \textit{A non-zero polynomial $f$ of degree $d$ over a commutitive ring $A$  is homogenous iff for every set of $n+1$ algebraically independent elements $u, t_1, \dots, t_n$} over $A$, 
$$f(ut_1, \dots, ut_n) = u^df(t_1, \dots, t_n).$$

\begin{proof}
    Let $f(X) = \sum_{(v)}a_{(v)}X_1^{v_1} \cdots X_n^{v_n}.$ If $f$ is homogenous of degree $d$, $v_1 + \cdots + v_n = d$ for all $a_{(v)} \neq 0$. $f(ut_1, \dots, ut_n) = \sum_{(v)} a_{(v)} (ut_1)^{v_1} \cdots (ut_n)^{v_n}$. Since $A$ is commutitive, this is equal to 
    $\sum_{(v)} a_{(v)} u^{v_1 + \dots + v_n}t_1^{v_1} \cdots t_n^{v_n}$.

    On the other hand, if $f(ut_1, \dots, ut_n) = u^df(t_1, \dots, t_n)$m, then $\sum_{(v)} a_{(v)}u^{v_1 + \cdots + v_n}  = f(u1_A, \dots, u1_A) = u^df(1_A, \dots, 1_A) = u^d\sum_{(v)} a_{(v)}.$ This is a polynomial in $u$ over $A$ and equality is assured iff $u^d = u^{v_1 + \cdots v_n}$. 


\end{proof}

Let $G$ be a \href{https://en.wikipedia.org/wiki/Monoid}{monid} and let $A[G]$ be the set of all mappings $\alpha: G \to A $ such that $\alpha(x) = 0$ for almost all $x \in G$. Addition is defined ordinarily and multiplication is defined as 
$$\alpha \beta (z) = \sum_{xy = z} \alpha(x) \beta(y).$$
Then $A[G]$ is a ring. A more convenient notation can be acheived if we define $a \cdot x$ as 
$$
    a \cdot x(z) = \begin{cases}
        a \text{ if } z = x\\
        0 \text{ if otherwise.}
\end{cases}$$

This way we can define, 
$\alpha = \sum_{x \in G} \alpha(x) \cdot x$, and 
$$\bigg(\sum_{x \in G} a_x \cdot x \bigg) \bigg(\sum_{y \in G} b_y \cdot y \bigg) = \bigg(\sum_{x, y} a_xb_y \cdot xy \bigg)$$

$$\bigg(\sum_{x \in G} a_x \cdot x \bigg) + \bigg(\sum_{x \in G} b_x \cdot y \bigg) = \bigg(\sum_{x\in G} (a_x + b_x) \cdot x \bigg), $$
where $\set{a_z}_{z \in G}$, $\set{b_z}_{z \in G}$ are the elements of $A$, most of them equal to $0$. 

The injective homomorphisms $x \mapsto 1_A \cdot x$ and $a \mapsto a \cdot e$ show that $G$ and $A$ are embedded in $A[G]$. \\ 
% 1.(xy) = (1.x)(1.y    )   


\noindent \textit{Let $A$ be a communtative ring and $S$ be a multiplicative subset \footnote{A subset containting $1_A$ and closed under multiplication}.  For $a, a' \in A$ and $s, s' \in S$, we say}
$$(a, s) \sim (a', s')$$
\textit{if there is $s_1 \in S$ such that}
$$s_1(as' - sa') = 0.$$
\textit{$\sim$ is an equivalance relation.}
\begin{proof}
    Symmetry and Reflexitvness are trivial. Transitivity can be verified as follows. Let $(a, b) \sim (c, d)$ and $(c, d) \sim (e, f)$. Then for some $s_1, s_2 \in S$, we have
    $$s_1ad  = s_1bc$$
    $$s_2de = s_2cf$$
Multiplying both sides of first and the second equation by $s_2f$ and $s_1b$, it follows that $(s_1s_2d)(af - be) = 0.$
\end{proof}
\noindent This construction of ring is called \textbf{ring of fraction of $A$ by $S$, $S^{-1}A$}. The homomorphism $A \mapsto S^{-1}A$ defined by $a \mapsto a/1_A$ is a universal object (See \Ref{universal_object}). If $A$ is an integral domain, then $S^{-1}A$ is the field of fractions.
\\

If $A$ has a unique maximal ideal, it is called \textbf{a local ring.} An intersting example is $A_{\I p} = S^{-1}A$,where $S$ is the multiplicative subset $A - \I p.$
\chapter{Modules}

The concept of rings is motivated by the properties of a set of \textit{endomorphims} on an (additive) abelian group. Left $R$-modules are the abelian groups $M$ such thaat there is a ring homomorphism $R \to \End(M)$.\\

\noindent \textit{Example:} If $J$ is an ideal of a ring $A$, then we can define an operation of an element $a, b \in A$ on $A/J$ as $a \cdot (x + J) \mapsto ax + J.$ This mapping is an endomorphim of $A/J$ because $a \cdot (x + y + J) = a \cdot (x + J) + a\cdot (y + J)$. We can define the a ring homomorphism from $A \to \End(A/J)$ trivially.  Therefore, $A$ defines a module structure over $A/J$.\\

To show a group $M$ is $A$-module, it suffieces to show that for $a, b \in A$, $x, y \in M$ 
$$a(x + y) = ax + ay \text{ and } (a + b)x = ax + by,$$
These conditions are equivalent to showing there is a ring homomorphism from the actions of elements of $A$ on $M$ to $\End(M)$.\\
\noindent\rule{\textwidth}{1pt}    
    \noindent Some basic constructions from the companion. Let $M$ be an $A$-module.
    \begin{enumerate}
        \item For $N \subseteqq M$, $\set{r \in A : rN = 0}$ forms an left ideal in $A$.
        \item For $N \leqq M$, $\set{r \in A: rM \subseteqq N}$ forms a right ideal of $A$.
        \item For $N \leqq M$, $\set{r \in A: rN \subseteqq N}$ forms a subring.
        \item If $N$ is a submodule, then the ideals in 1 and 2 are $2$-sided. Here, it is important to point out that when $N$ is a submodule, then closure of the actions of $A$ on $N$ is maintained.
    \end{enumerate}
    If $x \in M$, then $Rx \iso R/I$, where $I$ is the annhilator ideal of $\{x\}$ as in 1. \\
    
    \noindent \textit{Every ideal (left, right and 2-sided) and subring of $A$ can be constructed in the above way}\\

    \noindent  
    \noindent\rule{\textwidth}{1pt}

    \begin{definition}
        A \textbf{module-homomorphism} is an additive group homomorphism $f: M \to M'$ from modules $M$ to module $M'$ and such that $f(ax) = af(x).$  
    \end{definition}

    \noindent \textit{If $f$ is module-homomorphism from $M$ to $M'$ then the kernel and the image of $f$ are submodules of $M$ and $M'$ respectively.}
    \begin{proof}
        Clearly, $\ker f \leqq M$ because $f$ is a group homomorphism. Let $a \in A$ and $x \in \ker f$. $f(ax) = af(x) = 0$. Hence, the kernel of $f$ is a submodule of $M$.

        Again, $\Img f \leqq M'$. $af(x) = f(ax) \in \Img f.$

     
    \end{proof}

    \horline\\
    $M'/f(M)$ is a universal(inital) among the modules $N$ with homomorphism $g:M' \to N$ such that $g\circ f = 0.$ That is the following diagram commutes and $\hat g$ is unique:
    \\
    \begin{center}
        
        \begin{tikzcd}[cramped]
            M && {M'} && N \\
            && {} \\
            && {M'/f(M)}
            \arrow["f"{description}, from=1-1, to=1-3]
            \arrow["g"{description}, from=1-3, to=1-5]
            \arrow["c"{description}, from=1-3, to=3-3]
            \arrow["{\hat{g}}"{description}, from=3-3, to=1-5]
        \end{tikzcd}
    \end{center}

    This is dual with the kernel of $f$ which is a terminal object among modules $N$ with homomorphism $g: N \to M$ such that $f \circ g = 0.$ Thus, it is called the \textbf{cokernel} of $f$.    
    \horline

    
    \begin{definition}
        A \textbf{monomorphism} is a module-homomorphism $u: N \to M$ characterized by the exact sequence $0 \to N \overset{u}{\to} M.$ Similarly, an \textbf{epimorphism} is characterized by dual exact sequence $N \overset{u}{\to} M \to 0.$ 
    \end{definition}

    These definitions concide with the definitions of one-to-one homomorphisms and surjective homomorphism in the category of modules over a ring $R$.

    \begin{definition}
        For a commutitive\footnote{the concept of algebras does not make much sense with non-commutative rings} ring $A$, we say $K$ is an $A$-algebra, if $K$ is a module with $E$ a \textbf{\href{https://en.wikipedia.org/wiki/Bilinear_map}{$A$-bilinear map}} $g:E \times E \to E.$ 
    \end{definition}

    \horline

    \noindent In the companion, the following remark is left.\\

    \textit{Let $A$ be a commutative ring. Then 
        $$\text{associative, unital }A\text{-algebra}\ R \equiv \text{Ring $R$ with a homomorphism $f:A \to Z(R).$}$$
    }
    \noindent $f$ is a way of endcoding the bilinear operator, and why it's into the center of $R$ is mainly because we require $a\cdot xy = (a\cdot x)y = x(a \cdot y) :=  f(a)xy = (f(a)x)y = xf(a)y $\\

    Another intersting remark is that algebras are abstractions of the natural structure of $A$-module-endomorphims of a module $M$, $\End_A(M)$, just like rings abstract the endomorphims of an abelian group.  \\
    \horline

    A sequence $\cdots \to A \overset{f}\to B \overset{g}\to C \to \cdots $ is called exact if $\Img f  = \ker g.$ We denote the group of of $A$-homomorphisms from $A$-module $X$ to $Y$ by $\Hom_A(X, Y).$
    \begin{proposition}
        Let $X, X', X''$ and $Y$ be $A$-modules. Then the short sequence
        $$X' \fto{\lambda} X \fto{\mu} X'' \to 0$$
        is exact if and only if 

        $$ \Hom_A(X', Y) \overset{\lambda'}\from \Hom_A(X, Y) \overset{\mu'}\from \Hom_A(X'', Y) \from 0$$
        is exact for all $Y$.
    \end{proposition}
    \begin{remark}
        This proposition is analogous to the duality of linear maps in vector spaces. 
    \end{remark}

    
    \begin{proof}
    Suppose the first sequence is exact. Then the following statements hold:
    \begin{enumerate}[(i)]
        \item $\Img \lambda = \ker \mu$ \label{item:1}
        \item $\Img \mu = X''$.\label{item:2}
    \end{enumerate}
    Let $g \mapsto g \circ \lambda = 0$. Since $\Img \lambda \subseteqq  \ker g,$ $g$ factors through $X/\Img \lambda.$ By \Ref{item:1} and \Ref{item:2}, $X/\Img \lambda \iso \Img \mu =  X''$ which implies $g = f \circ \mu$ for some $f \in \Hom (X'', Y)$. This shows $\ker \lambda' \subseteqq \Img \mu'.$ Similarly, let  $h \circ \mu \in \Img \mu'.$ By \Ref{item:1}, the composition of this with $\lambda$, $h \circ \mu \circ \lambda = 0$, implying $\Img \mu' \subseteqq \ker \lambda'$ (thus $\Img \mu' = \ker \lambda'$).
    The first implication of the proposition follows from the fact that if $f \mapsto f \circ \mu = 0$ for some $f: X'' \to Y$, then $f = 0$  by \Ref{item:2}.
    \\

    The proof of the converse is an easy application of the following common technique: \textbf{To study the consequences of a condition holding for all morphisms of a given sort, consider a universal example.}

    Suppose the second sequence is exact, i.e.,
    \begin{enumerate}[(i)]
        \item $\ker \lambda'  =  \Img \mu'$ \label{item:3}
        \item $\ker \mu' = 0$. \label{item:4}
    \end{enumerate}

    By \Ref{item:3},  $\ker \lambda' \supseteqq  \Img \mu'.$ That is,  for every $Y$ and $f:X'' \to Y$ $f \circ \mu \circ \lambda = 0.$ Now, consider the universal example for all $f$s, i.e.,  the category of morphisms from $X''$ which is $\id$, the identity morphisms. $\id \circ \mu \circ \lambda  = \mu \circ \lambda = 0$ implies $\ker \mu \supseteqq \Img \lambda.$ 

    Similarly, the condition $\ker \lambda' \subseteqq \Img \mu' $ implies for every $Y$, a map $g: X \to Y$ such that $g \circ \lambda  = 0$  can be factored  through $X''.$ The universal object of all morphisms from $X \to Y$ which are $0$ at $\Img \lambda$ is the canonical homomorphism $q: X \to X/\Img \lambda.$ Hence $q = f \circ \mu$ which is obviously $0$ on $\Img \lambda$ and thus $\ker q = \Img \lambda \supseteqq \ker \mu.$
       
    Finally, the universal object of morphisms from $X'' \to Y$  annihilated by $\Img \mu$ is the canonical morphism $p:X'' \to X''/\Img \mu.$ However, \Ref{item:4} implies $p = 0$ and $X'' \iso \Img \mu$ which completes the proof.
    \end{proof}

    Let $\set{M}_{i \in I}$ be a family of submodules of $M$. Then we have the induced homomorphism 
    $$\lambda_*: \bigoplus_{i \in I} M_i \to M$$
    defind by $\lambda_*((x_i)) = \sum  x_i$. If $\lambda_*$ is isomporphism, then we call the family $\set{M}_{i \in I}$, \textbf{direct sum decomposition} of $M$ as we have

    $$\bigoplus M_i = M.$$
    Otherwise, if $\lambda_*$ is only surjective, we can write
    $$M = \sum M_i$$
    \begin{remark}
        This notion is analogous to linear independece and direct sums in linear algebra.
    \end{remark}


    Let $M_1, M_2, N$ be modules. Then we have the following isomorphism of abelian groups

    $$\Hom(M_1 \oplus M_2, N) \iso \Hom(M_1, N) \times \Hom(M_2, N)$$
    $$\Hom(N, M_1 \times M_2) \iso \Hom(N, M_1) \times \Hom(N, M_2)$$

    The first isomorphism follows from the association $f \mapsto (f_1, f_2)$ where $f$ is an element of the LHS group and $f_i: M_i \to N$ are the homomorphisms defined by $f_i = f \circ I_i$. The second one follows with similar associations.


    \begin{proposition}
        Let the following sequence of modules be exact:
        $$0 \to M' \fto{f} M \fto{g} M'' \to 0$$
        The following conditions are equivalent
        \begin{enumerate}
            \item There is a homomorphism $\varphi: M'' \to M$ such that $\id = g \circ \varphi.$
            \item There is a homomorphism $\psi: M \to M'$ such that $\id = \psi \circ f.$
        \end{enumerate}
        If these conditions are satisfied, then we have the following isomporphisms:
        $$M = \ker g \oplus \Img \varphi = \ker \psi \oplus \Img f \iso M' \oplus M''.$$
    \end{proposition}

    The general idea is the exactness of the sequence makes $M$ factorize into $M' \times M/M'$ in group theory terms.
    \begin{proof}
        Let $x \in M$. Then $x - \varphi(g(x)) \in \ker g$ by definition. Thus $x = (x - \varphi(g(x))) + \varphi(g(x)) \in \ker g + \Img \varphi$. This sum is direct because $\ker g \cap \Img \varphi = 0.$ The others isomorphisms follow immediately.
    \end{proof}


    \begin{definition}
        A \textbf{free module} is an $A$-module that admits a basis. 
    \end{definition}

    \begin{proposition}
        Let $M$ be a free module with basis $\set{x_i}_{i \in I}$ and let $\I a$ be a two-sided ideal of $A$. Then 
        \begin{enumerate} 
        \item $\I a M$ is also a submodule of $M$ that is also $\I a$-module.
        \item Each $\I a x_i$ a submodule of $Ax_i$.
        \item We have the module isomorphism 
        $$M/\I a  M \iso  \bigoplus_{i \in I} Ax_i/\I a x_i.$$
        \item $Ax_i/\I ax_i$ is isomorphic to $A/\I a$ as $A$-module
        \item Suppose $A$ is commutative. Then $A/\I a$ is a ring. Furthermore $M/\I aM$ is a free over $A/\I a$ and $Ax_i/\I ax_i$ is a free over $A/\I a$. If $\overline{x_i}$ is the image of $x_i$ under the canonical homomorphism $Ax_i \to Ax_i/\I a x_i$, then $\overline{x_i}$ is the basis of $Ax_i/\I ax_i$.
    \end{enumerate}
    \end{proposition}
    \begin{proof}
        We go through the statements one by one:
        \begin{enumerate}
            \item Let $x \in M$. Then $x = \sum_{i \in I} a_ix_i$ uniquely for $\set{a_i}_{i\in I} \subseteqq A$. By definition, $\I a M = \set{\sum yx : y \in \I a, x  \in M}$. But $yx = \sum_i ya_ix_i = \sum_i y_ix_i \in M$ where $y_i \in \I a$ because $\I a$ is two-sided ideal.
            
            \item Clearly, $\I a x_i \subseteqq Ax_i$. Let $a', b' \in \I a$ and $a, b, c \in A$.  $Ax_i$ is a $A$-module because $(a + b)cx_i = (ac + bc)x_i = acx_i + bcx_i$ and $c(a'x_i + b'x_i) = c(a + b)x_i = (ca + cb)x_i = cax_i + cbx_i.$ The statement follows from $A\I a x_i \subseteqq \I a x_i$
            
            \item By definition, $M = \bigoplus_{i \in I} Ax_i$. Consider the isomprohism 
            $$\sum_{i \in I} a_ix_i \mapsto (a_ix_i)_{i \in I}$$
            which induces the isomorphism
            $$\sum_{i \in I} a_ix_i + \I a M \mapsto  (a_ix_i + \I aM)_{i \in I}.$$
            Since $\I aM$ is a $\I a$-module and  $a_ix_i+ \I  aM = a_ix_i + \I a x_i$, and $Ax_i/\I a x_i$ is an $A/\I a$-module, the statement is true.

            \item Consider the isomoprhism $1_A \mapsto  x_i.$
            \item $A/\I a$ is a ring of cosets of $\I a$.  $M/\I  aM$ is free as the basis $\set{x_i}_{i \in I}$,  serves as a basis for $M/\I aM$ over $A/\I a$.                
        \end{enumerate}
    \end{proof}

     We say an exact sequence of abelian groups
    $$0 \to A \to B \to C \to 0$$
    splits if $B \iso A \oplus C.$

    \noindent \textit{For Example: } The sequence
    $$0 \to \C{2} \fto{x \mapsto x} \C{2} \times \C{2} \fto{-1} \C{2} \to 0 $$
    splits but
    $$0  \to \C{2} \fto{x \mapsto 2x} \C{4} \fto{\mod 2} \C{2} \to 0$$
    does not split.

    \begin{proposition}
        Every surjective module-homomorphism from a $A$-module, $M$ to a \textbf{free} a $A$-module $F$ splits.  
    \end{proposition}
    \begin{proof}
        Let $\phi: M \to F$ be a surjective homomorphism. By the first isomorphism theorem, $F \iso M/\ker \phi$. Let $\set{x_i + \ker \phi}_{i \in I}$ form the basis of $M/\ker \phi$. Define $\psi: M/\ker \phi \to M$ as
        $$\psi\bigg(\sum_{i \in I} a_i x_i + \ker \phi \bigg)= \sum_{i \in I} a_ix_i.$$ 
        Clearly $\phi \circ \psi = \id$
    \end{proof}

    $F$ need not be a free module for $A \to F$ to split. Modules that admit splitting like the above are called \textbf{projective}. Here are four equivalent conditions that are satisfied by a projective module $P$:

    \begin{enumerate}
        \item Given a homomorphism $f: P \to M$ and a subjective homomorphism $g: M' \to M$, there exists a homomorphism $h: P \to M'$ that makes the following diagram commute:\\
        % https://q.uiver.app/#q=WzAsNCxbMSwwLCJQIl0sWzAsMSwiTSciXSxbMSwxLCJNIl0sWzIsMSwiMCJdLFswLDIsImYiXSxbMCwxLCJoIiwyXSxbMSwyLCJnIiwyXSxbMiwzXV0=
       \begin{center}
         \begin{tikzcd}[cramped]
            & P \\
            {M'} & M & 0
            \arrow["h"', from=1-2, to=2-1]
            \arrow["f", from=1-2, to=2-2]
            \arrow["g"', from=2-1, to=2-2]
            \arrow[from=2-2, to=2-3]
        \end{tikzcd}
       \end{center}
        \item The exact sequence $0 \to M' \to M \to P \to 0 $ splits
        \item There exists a module $M$ such that $P \oplus M$ is free.
        \item The functor $M \mapsto \Hom_A(P, M)$ is exact.
    \end{enumerate}

    \begin{proof}
        We only leave the proof of $(4) \implies (1)$ as the rest is found in the book. Consider (4) is true, i.e, if $0 \to M'' \to M' \fto{g} M \to 0$ is exact, $0 \to \Hom_A(P, M'') \to \Hom_A(P, M') \fto{\lambda} \Hom_A(P, M) \to 0$ is also exact. Since $\lambda$ is surjective,  for any $f \in \Hom_A(P, M)$, we can find $h \in \Hom_A(P, M')$ such that $\lambda(h) = g \circ h = f$.  
    \end{proof}

    \begin{proposition}
        Let $V$ be a vector space. Let $\Gamma$ be the set of generators of $V$ and $S$ be a set of any linearly independent elements. Then, there is a basis $\m B$ such that $S \subseteqq \m B \subseteqq \Gamma$.  
    \end{proposition}

    \begin{proof}
        Let $\m I$ be the sets $T \supseteqq S$ that are linearly independent. Assuming $V \neq \set{0}$, $\m I$ is non-empty. Clearly $\m I$ is a poset by ascending inclusion. Since if $T_i \subseteqq T_{i+1} \in \m I$ then $T_i \cup T_{i+1}$ is linearly independent making $\m I$ an inductively orderd set. By zorns lemma, there is a maximal element of $\m I$. Let's call that $\m B$ and let $\gen{B} = W$. If $W \neq V$, then there is $x \in V$ such that $x \neq \sum_{y \in \m B} a_y y$ making $\m B \cup x$ linearly independent and contradictng maximality of $\m B.$ Thus $V = W.$    
    \end{proof}


    \begin{proposition}
        Let $V, U$ be  vector spaces over field $K$ and let $V \fto{f} U $ be homomorphism. Then we have
        $$\dim_K V = \dim_K \ker f + \dim_K \Img f.$$ 
    \end{proposition}

    \begin{proof}
        Let $\set{w_i}_{i \in I}$ and $\set{u_i}_{i \in I}$ be the basis of $\ker f$ and $\Img f$ resp. Let $\set{v_i}_{i \in I}$ be a family of elements such that $f(v_i) = u_i.$  Let $x \in V$. Then we have,
        $$f(x) = \sum_{i \in I} a_i u_i,$$
        where $\set{a_i}_{i \in I}$ is a family in $K$ such that all except finit of them are $0$. This implies, 
        $$y = x - \sum_{i \in I} a_iv_i \in \ker f.$$
        However, $\ker f$ is a vector field and $y = \sum_i b_iw_i.$ This implies 
        $$x = \sum_{i \in I} a_i v_i + \sum_{j \in I} b_j w_j.$$
        Proving $\set{v_i, w_i}_{i \in I }$ generates $V$. It remains to show that this generator is linearly indepenendent.

        Let $0 = \sum_{i \in I} a_i v_i + \sum_{j \in I} b_j w_j.$ Then $f(\sum_{i \in I} a_i v_i + \sum_{j \in I} b_j w_j) = 0 + \sum_{i \in I} a_i f(v_i) + 0 = \sum_{i \in I} a_i u_i = 0 \implies a_i = 0 \implies b_j = 0.$
    \end{proof}

    \horline  

    \noindent An important insight from the companion:
    \textit{A free left $R$-module with rank $n$ is isomorphic to a standard\footnote{By standard, we mean where the action of $R$ is trivial as in linear algebra} module $R^n.$} This helps us derive the following facts about modules over non-field ring:

    \begin{itemize}
        \item \textit{If $R \fto{f} S$ is a homomorphism and $m, n$ are positive integers such that $R^m \iso R^n$, then $S^m \iso S^n$.}\\
        If $\mathcal{M}$ is a (isomorphic) transformation from $R^m \to R^n$, then is $f(\mathcal{M})$ is too from $S^m \to S^n$.

        \item \textit{If there is a homomorphism onto a  field (division ring), then all left $R$-modules have a fixed number of elements in their basis.}
    
        
        This follows by taking $f = R \mapsto R/I$ where $I$ is a maximal ideal.
        
        
    \end{itemize}

    \textbf{Warning:} Modules over non-commutative  rings do not  necessarily have unique ranks. 

    \subsection*{Dual Space and Dual Module}
    Let $E$ be a free module over a commutitive ring $A$. We denote the \textbf{dual module}, $\Hom_A(E, A)$, of $E$ by $\dual{E}$ and we call the elements of $\dual{E}$ as \textbf{functionals.}

    If $x \in E$, then $x$ induces a map $\gen{x, -}$ from $\dual{E}$ to itself defined by $\gen{x, f} = f(x).$ 

    \horline
    The map $\theta: E \to \ddual{E}$ is not surjective for the following reason. In infinite-dimensional modules over a field $A$, $\ddual{E}$ is also infinite dimensional. However, $x$ can be expressed as a linear combination of the basis of $E$ and so is $\theta(x).$\\
    \horline

    \begin{proposition}
        If $E$ is free, so is $\dual{E}.$ Moreover, $\rank E = \rank \dual{E}$
    \end{proposition}

    \begin{theorem}
        Let $E$ be finite dimensional. The map $x \fto{\phi} (f \mapsto \gen{x, f})$ is an isomorphism from $E$ to $\ddual{E}.$
    \end{theorem}

    \begin{theorem}

        Let $U, V, W$ be finite-dimensional free modules over commutative ring $A$. If the sequence
        $$0 \to W \to V \to U \to 0$$
        is exact, then so is
        $$ 0 \to \dual{U} \to \dual{V} \to \dual{W}.$$
    \end{theorem}

    \horline
    \noindent \textit{Why it is called a sequence splits?}
    A short sequence
    $$0 \to C \to C \oplus B \fto{g} B \to  A$$
    is \textbf{splits} into

    $$0 \to C \to C \to 0 \to 0$$
    $$0 \to 0 \to B \to B \to 0$$

    We require a right inverse map $g'$,i.e., that satisfies $\id = g \circ g': B \to C \oplus B$, to say so, the action of this map on $C$ would be $0$ and the action on $B$ would be $g^{-1}$ \\
    \horline

    \subsection*{Modules over Principal Ideal Domains}
    \begin{theorem}
        Let $R$ be a principal ideal domain and let $F$ be a free $R$-module. If $M$ is a submodule of $F$, then $M$ is free with rank less than or equal to $\rank F$. 
    \end{theorem}
    \begin{proof}[sketch]
        Let $M_i$ be the submodule of $M$ generated by the basis subset $\{x_1, \dots ,x_i\}$. Let $\I a_{i+1}$ denote the set of coffecients of $x_{i+1}$ in $M - M_i.$ If $\I a_{i + 1} = 0$, we are done. If not, observe that $RM_i \subseteqq M_i$ and $\I a_{i+1} = \gen{a_{i+1}}$ for some $a_{i+1} \in  R.$ Let $w := \sum_{j \leqq i} b_jx_j + a_{i+1}x_{i+1}$. Then $M_{i+1} = M_i + Rw.$
    \end{proof}
    \begin{remark}
        The PID nature of $R$ permits the constructions of  \textit{generators} $w_i$ of $M$ corresponsing   to the generators $x_i$
    \end{remark}

    \noindent \textbf{NB:} Finitely generated modules are factor modules of a free module.

    \begin{definition}
        An $R$-module $M$ is called a \textbf{torsion} module if for some $x \in M$, there is an element $a \in R$ such that $ax = 0$. We denote the module that contain all torsion elements by $\tor M.$
    \end{definition}

    \begin{theorem}
        Let $E$ be finitely generated. The factor module $E/\tor E$ is free and there is a free submodule $F$ of $E$ such that 
        $$E = \tor E \oplus F.$$
    \end{theorem}   

    Modules ove PID exhibit similar characterstics as abelian groups. For example, the cyclic $p$-groups are analogous to a moule generated by an element $x$ modulo a prime ideal, i.e $Rx/(p)x$. We call a module of type $(r_1, \dots, r_k)$ if is a product of modules isomorphic to $R/(p^{r_i}).$ The following two theorems support the similarity even more by stating the equivalent statements to the fundamental theorem of abelian groups.
    
    \begin{theorem}
        Let $R$ be a princial ideal domain and let $E$ be a finitely generated torsion 
        module over $R$. Let $E(p)$ denote all elements of $E$ with exponent\footnote{An exponent of a module $M$ (an element of a module $x$ resp.) is an element $m$ of $R$ such that $mE$ (resp. $mx$) is 0.} that is a power of a prime element $p \in R$. Then $E$ has the decomposition

        $$E = \bigoplus_p E(p), $$
        where the direct sum is over $p$ such that $E(p) \neq 0.$ Moreover, for each $p$, we have
        
        $$E(p) = R/(p^{v_1}) \oplus \cdots \oplus R/(p^{v_r})$$
        with $1 \leqq v_1  \leqq \cdots \leqq v_r$ that are determined uniquely.
    \end{theorem}
    \noindent $E_m := $ the kernel of the map $x \mapsto mx$ in $E$.
    \begin{proof}
        Let $a$ be an exponent of $E$. Consider the map $x \mapsto ax.$ Let $a = bc$ with $(b,c) = (1)$. Let $xb + yc = 1$. Then $v = vxb + vyc$ where $vxb \in E_c$ and $vyc \in E_b$. Moreover, $E_b \cap E_c = 0.$ Thus $E_a = E_b \oplus E_c.$ By induction, the stated decomposition of $E$ follows. 

        Next, we show that $E(p)$ is a direct sum as stated. 
        \\\horline The intuition for such decomposition of $E(p)$ comes from boxing all elements of $E(p)$ with the same period\footnote{A period $T$ of an element $x$ is an element of $R$ such that the kernel of the mao $a \mapsto ax $ equals $\gen{T}$} into a direct summand.\\
        \horline
        We will use induction. Consider the canonical map from $E(p) \to E(p)/(x)$ where $x$ is an element of $E(p)$ with maximal period, $p^r$. Suppose $\set{\overline{y_1}, \dots, \overline{y_m}}$ are independent\footnote{We call a family of elements $\set{y_i}$ of a module $M$ independent if $\sum_i a_iy_i = 0 \iff a_iy_i = 0\ \forall i$} elements of $E(p)/(x)$ with representatives $\set{y_1,  \dots, y_m}$ in $E(p)$. If $p^{n_i}$ is the period of $\overline{y_i}$, then $p^{n_i}y_i = p^scx$ for some $c \in R$, $p \nmid c.$ By assumption, $r \geqq s$, thus $p^{n_i - s  + r} = 0 \implies n_i - s + r \leqq r \implies n_i \leqq s.$ Therefore the element $y_i - p^{s - n}cx$ is well-defined and has period equal to that of $\overline{y_i}.$

        Moreover the set $\set{x, y_1, \dots, y_m}$ is independent because if $bx + \sum_ia_iy_i  = 0$, then $\sum_i a_i\overline{y_i} = 0$ which can not happen unless $a_i\overline{y_i} = 0$ for all $i$. But by previous part of the proof, this implies all period $c_i \mid a_i \implies a_iy_i = 0$ and $bx = 0.$  
        
        Thus, $E(p)$ has $m+1$ independent elements $x, y_1 \dots, y_m$. It is clear that $(x, y_1, \dots, y_m) = (x)\oplus(y_1)\oplus\cdots\oplus(y_m)$ by independece. Note that if $w \in E$ has period $t$, then $(w) \iso R/\gen{t}.$ This proves the existence of such decomposition.

        Uniquness of the decomposition follows as following. Let $(s_1, \dots, s_m)$ and $(r_1, \dots, r_n)$ be two types of $E(p)$ with $s_i \leqq s_{i+1}$ and $r_i \leqq r_{i+1}$. WLOG, let $s_i < r_i$ be the first different entries. Clearly, there is an element $x \in E(p)$ with period $p^{s_i}$. However, no such element exist in $R/(p^{r_{i}}) \oplus \cdots \oplus R/(p^{r_n}).$ Thus $s_i = r_i$.   
    \end{proof}


    \begin{remark}
        The proof of theorem 7.8 on the book utilizes a trick to select a basis set with particular property. The trick relies (generally speaking) on the fact that functionals capture the properties of basis.

        \noindent \textit{For example}: The dimenstion of a free module $M$ is equal to $\max_{\lambda \in \dual{M}} \dim \lambda(M).$
    \end{remark}


      \subsection*{Direct and Inverse Limits}
    Let $I$ be a \href{https://en.wikipedia.org/wiki/Directed_set}{directed set}. 
    Let $\set{A_i}_{i \in I}$ be a family of $A$-modules and let$\set{f_{i,j}:A_i \to A_j}$ be a family of $A$-homomorphism satisfying
    $$
        f_{i, i} = \id$$
        $$f_{i, k} = f_{j, k} \circ f_{i, j} \text{ if } i < j. 
    $$
    We call this family of morphisms, \textbf{a directed family of morphisms}. When we have a family like $\set{A_i}$, we want to study their properties together. The \textbf{direct limit} has the required algebraic properties to do so and it's defined as follows.

    Consturct a category $\mathcal{C}$ by defining $\Ob(\mathcal{C})$ as the pair $(A, f_i)$ with $A$ in the family of modules and $f_i: A_i \to A$ that makes the following diagram commute

    \begin{center}
    \begin{tikzcd}
    A_i \arrow[dr, "f_{i,j}"'] \arrow[rr, "f_i"] & & A \\
    & A_j \arrow[ur, "f_j"'] &
    \end{tikzcd}
    \end{center}

    where the morphisms are $f_i$ themselves. The direct limit $(B = \dirlim A_i, h_i)$ is the universal object of this category, i.e., for every $(C, g_i)$ in this cateory there is a unique homomorphism $t$ that makes the following diagram commute


    % \noindent \textit{There is a unique homomorphism from the basis of a module $M$ to any subset of another module $N$.}
    % ker + F
    % A/ker f is free  x ... x + ker f, A = u + x ... x  ker + A
    % a = ker f + \sum xi
    % Z/4 , Z/4/Z/2 Z/2
    % 
% \noindent The reason why $f$ is into the center of $R$ in particular is closly tied to the identity $a \cdot x = (a \cdot 1_R)x$.

% A/a xi iso Axi / axi....   Axi -> (A/a)xi  f(aixi) = (ai mod a) xi   
    % ra in I
    % M = A
    % N = {r in A such that gr = 0 for all g in S} if I = <S>
    % N = <S> rm
    % r.m = sum gi.ai 
    % N  = {r in A such that }
    
    

\chapter{Polynomials}

\begin{proposition}

  If $k$ is a field and $k[X_1, \dots, X_m]$ is the ring of polynomials over the variables $X_1, \dots, X_n$. Let $f \in k[X_1, \cdots, X_m]$ and $S_1, \cdots,  S_m$ be infinites subsets of the field $k$ such that $f(a_1, \cdots, a_n)$ for all $a_i \in S_i$. Then $f = 0$.

\end{proposition}
\begin{proof}

  For $m = 1$, the propostion is trivial. For $m > 1$, note that
  $$k[X_1, \cdots, X_m] = k[X_1, \cdots, X_{m-1}][X_m].$$

  For fixed $a_1, \dots, a_{m-1}$, $f(a_1, \cdots a_{m-1}, X_n)$ is then $\in k[X_n]$ and thus $f(a_1, \cdots, a_{m-1}, X_n) = 0$ obtainig the result by symmetry and induction.

\end{proof}

\begin{theorem}
  Let $k$ be a field and let $U$ be a finite multiplicative subgroup $k$. Then $U$ is cyclic.
\end{theorem}

\begin{proof}
  Let $U = \prod_p U(p)$ where $U(p)$ is a $p$-group for each prime $p$. Let $a \in U(p)$ be an element with maximal power say $p^r$. Then for all $b \in U(p)$,  $b^{p^r} - 1 = 0$ making $|U(p)| \leq p^r$. Hence each $U(p)$ is cyclic.

\end{proof}
\begin{remark}
  Generally, certain polyonmials over fields like $X^{p^r} - 1$ in the above proof, help us enumerate elements of the field with certain characterstics by measns of their roots.
\end{remark}


\begin{definition}[Algebraic Closure]
  A field $k$ is called algebaically closed if all polyonmials in $k[X]$ of degree $\geq 1$ have all their roots in $k$.
\end{definition}

\begin{definition}[Frobenius Map]
  If $k$ is a field with characteristic $p$, we call the map
  $$x \mapsto x^{p^r}$$
  the frobenius map or frobenius endomorphism
\end{definition}

\subsection*{Polynomials over a Factorial Ring}
Let $A$ be a factorial ring and $K$ be its field of fraction.
\newcommand{\ord}{\mathrm{ord}}
\begin{definition}[Order]
  If $a \in K$ and $p \in A$ be a prime element.
  $$\ord_p: K \to \ZZ,$$
  $$\ord_p(a) := r: a = p^rx/y, p \nmid x, p \nmid y.$$

  If $f \in K[X]$, $f(X) = \sum a_i x^i$, we extend the above definition as
  $$\ord_p f = \min_p \ord_p (a_i),$$
  where the minimum is taken over all primes $p$ of $A$.
\end{definition}

\newcommand{\cont}{\mathrm{cont}}
\begin{definition}[$p$-content, content]
  We say the element $up^{\ord_p f}$, a $p$-content for $f$ for any unit $u$. Then the content of $f$, denoted by $\cont(f)$ is defined as
  $$ \prod_p p^{\ord_p f},$$
  over all primes $p$, upto multiplication by a unit.
\end{definition}

\begin{remark}
  Content is a generalization of the concept of gcd for fractions. For instance, $\cont(p) = p$, $\cont(px + q) = 1$, $\cont(px + p) = p$ for prime $p, q$.
\end{remark}


\begin{theorem}[Gauss Lemma]
  For any two $f, g \in K[X]$, we have
  $$\cont(fg) = \cont(f)\cont(g).$$
  \end{theorem}

  \noindent \textit{Sketch 1.}: If both $f$ and $g$ are primitive, then $fg$ is primitive. This can be shown by noting that for any prime $p$, if we can not extract $p$ from both $f$ and $g$, then there is a coefficient in $fg$ namely $c = \sum_{i+j = r + s} a_i b_j$ where $r$ and $s$ are the largest integers (resp) such that  $a_r$ and $b_s$ are indivisible by $p$ and $c$ is thus indivisible by $p$.


  \noindent \textit{Sketch 2.}
  Considering the reduction modulo a prime $p$ of two polynomials $f$, $g$, say $\bar f$ and $\bar g$, we have
  $$\bar{fg} = \bar f \bar g.$$
  Since $A/(p)$ is an integral domain, $\overline{fg} = 0 \iff \overline f = \overline g= 0$.

  \begin{theorem}
    $A[X]$ is factorial and the primes are primes of $A$ or irreducible polynomials of $K[X]$ with content of $1$.
  \end{theorem}

  \begin{proof}
    Let $f$ factorize as follows in $K[X]$

    $$f(X) = c \prod_i p_i(X),$$

    such that $\cont(p_i) = 1$. Since $\cont(f) = c$, $c \in A$ and there exists a factorization of $f$ in $A[X]$. Uniquness follows from uniquness of factorization in $K[X]$ upto multiplication by units and unitary content of irreducibles in $A$.
  \end{proof}


  \subsection*{Critera of Irreducibility}
  \begin{theorem}[Eisenstein's Criterion of Irreducibility]
    Let $A$ be a factorial ring and let $f \in A[X]$ such that
    $$f(X) = a_0 + a_1x + \cdots a_nx^n.$$
    Let $p$ be a prime in $A$. If we have
    \begin{eqnarray*}
      a_n \nequiv 0 \mod p & a_i \equiv 0 \mod p &  i < n \\
                           & a_0 \nequiv 0 \mod p^2 &       
    \end{eqnarray*}
    then $f$ is irreducible in $A[X]$ (thus $K[X]$).
    
  \end{theorem}

  \noindent \textit{Sketch}: If $f$ were reducible to $g, h$ such that $[X^n]g = b_n$, $[X^n]h = c_n$ and $\deg g = m$, $\deg h = n$  then neither of $b_m$ and $c_n$ are divisible by $p$. Moreover, WLOG, there is greaterst index $r$ such that all of $c_i, i > r$ are divisible by $p$, then
  $$[X^r]f = b_0c_r + \cdots $$
  is not divisible by $p$.

  \begin{theorem}[Reduction Criterion]
    Let $A$, $B$ be entire rings and let $\phi: A \to B$ be a homomorphism. Let $K$, $L$ be the quotient fields of $A$, $B$ resp. Assume for $f \in A[X]$,  $\phi f \neq 0$ and $\deg \phi f =  \deg f$. If $\phi f$ is irreducible in $L[X]$, then $f$ does not factorize to $g, h \in A[X]$ such that both $\deg g$, $\deg h \geq 1$.
    
  \end{theorem}

  \begin{proof}
    Since $\phi f = (\phi g) (\phi h)$, by irreduciblity of $\phi f$, one of the two factors on the right should have degree $0$. But $\deg \phi f = \deg \phi g + \deg \phi h$ by assumption, thus $f = c \cdot h$ for some $c \in A$.
  \end{proof}
  \begin{remark}
    
    This theorem is powerful test to check irreducibility. Eg. $X^p - X - 1$ is irreducibile over the field $\ZZ/p\ZZ$ thus irreducible over $\mathbb{Q}$.
  \end{remark}


  \subsection*{Hilbert's Theorem}

  \begin{theorem}[Hilbert's Theorem]
    If $A$ is commutitive and Noetherian, so is $A[X]$.
  \end{theorem}

  \noindent \textit{Sketch}
    Take an ideal of $A[X]$, $\I U = \oplus \I a_i X^i$. By ACC, there is $r$ such that $\I a_r = \I a_{r + s}$. Since $\I a_i$ is finitely generated, say by $a^i_j$, for $0  \leq i \leq r$, there are polynomials $f_{ij}(X) = a_{j}^{i}X^{i} + g(X)$, $g \in \I U$, $\deg g < i$, that generate $A[X]$ and the number of $f_{ij}$ is finite.


    \subsection*{Partial Fractions}

    \newcommand*{\Frac}{\mathrm{frac}}
    \begin{theorem}
      Let $A$ be a principal entire ring and let $K = \Frac(A)$. Let $\alpha \in K$ and $P$ be the set of representatives of the irreducibles of $A$, i.e, unique upto multiplication by units of $A$. For each $p \in P$, there exists an element $\alpha_p$ and non-negative integer $j(p)$ with $\gcd(p^{j(p)}, \alpha_p) = 1$ that satisfies

      $$\alpha = \sum_{p \in P} \dfrac{\alpha_p}{p^{j(p)}}$$

      with $j(p) = 0$ for all but finite elements of $P$. Moreover, this expression is unique upto the condition $\alpha_p \equiv \alpha_p' \mod p^{j(p)}$.
    \end{theorem}

    \begin{theorem}
      Let $k$ be a field and $k[X]$ be the ring of polynomials over $k$. Let $f, g \in k[X]$ such that $\deg g \geq 1$. There exists a unique sequence of polyonmials $f_0, \dots, f_d$ with $\deg f_i < \deg g$ such that

      $$f = f_0 + f_1g + \cdots f_dg^d.$$
    \end{theorem}

    The expresion of $f$ as such is called the \textbf{$g$-adic expansion} of $f$


    \subsection*{Symmetric Polynomials}

    Define the monomials $s_i$ as follows:
    $$\prod_{i=1}^n(X + t_i)  = \sum_{i= 0}^n s_i X^{n-i}$$
    \begin{theorem}
      Let $f(t) \in A[t_1, \dots, t_n]$ be a symmetric polyomial with degree $d$. Then There is polynomial $g$ of weight $\leq d$ such that $f(t_1, \dots, t_n) = g(s_1, \dots, s_n)$. 
    \end{theorem}


    \subsection*{Mason-Stothers Theorem and The $abc$ Conjecture}
    Let $n_0(f)$ be the number of distinct roots of the polynomisl $f \in K[X]$.
    \begin{theorem}[Mason-Stothers]
      If $a, b \in  K[t]$ are relatively prime polyonmials in an algebraically closed field $K$, then
      $$\max(\deg(a, b)) \leq n_o(ab(a + b)) - 1.$$
    \end{theorem}


    \begin{conjecture}[$abc$ conjecture]
      For a given $\epsilon > 0$, relatively prime integers $a, b$ and their sum $c$, and a constant factor $C(\epsilon)$ depending only on $\epsilon$,

      $$\max(|a|, |b|, |c|) \leq C(\epsilon)N_0(abc)^{1 + \epsilon},$$

      where $N_0(x)$ is the product of distinct prime divisors of $x$, called radical of $x$. 
      
    \end{conjecture}

    \subsection*{The Resultant}
    
    Let $v = (v_0, \dots, v_n)$ and $w = (w_0, \dots, w_m)$ be algebraically independent
    over a commutitive ring $A$. Let
    
$$f_v(X) = \sum_{i=0^n} v_iX^i, \quad g_w = \sum_{i = 0}^{m} w_i X^i$$
\[
\operatorname{Res}(f_v,g_w)
= \det
\begin{bmatrix}
v_0 & v_1 & \cdots & v_n & 0 & \cdots & 0 \\
0 & v_0 & v_1 & \cdots & v_n & \cdots & 0 \\
\vdots & & \ddots & & & \ddots & \vdots \\
0 & \cdots & 0 & v_0 & v_1 & \cdots & v_n \\[6pt]
w_0 & w_1 & \cdots & w_m & 0 & \cdots & 0 \\
0 & w_0 & w_1 & \cdots & w_m & \cdots & 0 \\
\vdots & & \ddots & & & \ddots & \vdots \\
0 & \cdots & 0 & w_0 & w_1 & \cdots & w_m
\end{bmatrix}
\]

\newcommand{\Res}{\operatorname{Res}}
Denote $R(v, w) = \Res(f_v, g_w)$. Then for any $z$, we have
$R(zv, w) = z^nR(v, w), \quad  R(v, zw) = z^mR(v, w).$
Hence, $R$ is homogenous in $v$ and $w$. We also have,
$$[v_0^mw_m^n]R(v, w) = 1.$$

One can also show there exists $\phi_{v, w}, \psi_{v, w} \in Z[v, w][X]$ such that

$$R(v, w) = \phi_{v, w}f_v + \psi_{v, w}g_w.$$
This relation serves as an 'invariant' (i.e. not depending on $X$). For example:

\begin{proposition}
For a subfield $K$ of $L$ and $f_a, g_b \in K[X]$ having a common root $\eta$, then $R(a, b) =  0$.
\end{proposition}

\begin{proposition}
  Let
  $$f_v(X) = v_0\prod_{i=1}^n (X - t_i) = \sum_{i=0}^n v_iX^i,$$
  $$g_w(X) = w_0\prod_{i=1}^n (X - u_i) = \sum_{i=0}^m w_iX^i.$$

  Then
  $$\Res(f_v, g_w) = v_0^mw_0^n \prod_{i=1}^m \prod_{j=1}^n (t_i - u_j).$$
  
\end{proposition}

\subsection*{Power Series}
\label{sec:power-series}
The formal power series $A[[X]]$ in one variable is formally defined as the ring of morphisms from $G$ to $A$ where $G$ is the multiplicative monioid of mappings from $\{X\} \to \mathbb{N}$. We denote an element $f$ as

$$f(X) = \sum_{n = 0}^\infty a_n X^n$$
where $f$ maps $(X^n: X \mapsto n)$ to $a_n \in  A.$

We define power series in $n$ variables $A[[X_1, \dots, X_n]]$ inductively.
If $k$ is a field then $k[[X_1, \cdots, X_n]]$ is a complete local ring where a sequence $\set{a_n}$ is considered Cauchy if there exist $N$ such that for all $n, m \geq N$, $a_n - a_m \in I^v$ for a given power $v$ and ideal $I$.

\horline\\
Here it's worth to consider what complete local ring means in other terms. A convergence point $a$ by the above notion is an element $a$ such that $a - a_k \in I^v$ for all $k \geq N(v)$ for any power $v$. This translates to an element $x = (x_0, \dots)$ in the projective limit $\invlim_n R/I^n$ such that $x_j = a_i \mod I^j$ for all $i \geq N(v)$ and $j \leq v$. Therefore, an element $x$ in the projective limit defines a convergence points for some Cauchy sequence and thus a ring is complete if $R$ is equal to the projective limit.

Locality, on the otherhand, implies $R$ has only one maximal ideal and anything outside the a given maximal ideal is invertible. 

\horline\\

\begin{theorem}
  Let $\I o$ be a complete local ring with maximal ideal $\I m$. Let $f(X) \in \I o[[X]]$ be given by
  $$f(X) = \sum_{i=0}^\infty a_iX^i$$
  such that not all $a_i$ lie in $\I m$. Suppose $a_0, \dots, a_{n-1} \in \I m$ and $a_n \in \I o^*$ is a unit. Then given $g \in \I o[[X]]$ one can solve the equation
  $$g = qf + r$$
  uniquely where $q \in \I o[[X]]$ and $r \in \I o[X]$, $\deg r \leq n-1.$ 
\end{theorem}

\begin{remark}
  If $a_0$ is a unit and all the other $a_i$ are non-units, $f$ is invertible and and one can always solve $g = qf$ by mutliplying inverse of $f$ to both sides. The theorem states a general case where $a_i \leq n-1$ are non-units, in which case one will have to make for the first $n-1$ terms of $g$ by adding polynomial $r$.
\end{remark}

\noindent \textit{Example}: Let $\I o = \ZZ$, and $f(X) = 2 + X + 2X^2 + 4X^3 + \cdots $. Let $g(X) = \sum_{n = 0}^\infty 2^n X^n.$ $f(X) = 1 + Xg(X) \implies (1 - 2X)f(X) = 1 - X \implies (1 + X + X^2 + \cdots )f(X) = g(X).$

The integrer $n$ is called \textbf{Weierstrass Degree} of $f$ and denoted $\deg_W(f)$.

\begin{theorem}[Weierstrass Preparation]
  Let $f$ be a polynomial in a complete local ring $\I o$ with $\deg_W(f) = n$. Then we can solve the following equation uniquely
  $$(X^n + b_{n-1}X^{n-1}+ \cdots + b_0)u = f(X),$$
  where $u$ is a unit in $\I o[[X]]$ and $b_i \in \I m.$
  
\end{theorem}
\begin{theorem}
 If $k$ is a field, then $k[[X_1, \dots, X_n]]$ is a UFD. 
\end{theorem}


\end{document}
