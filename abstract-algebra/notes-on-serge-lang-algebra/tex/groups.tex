\chapter{Groups}

\begin{theorem}[Sylow Theorems]
    Let $G$ be a finite group with $p$ divides $\size{G}$, where $p$ is a prime. Then
    \begin{enumerate}
        \item There exists a Sylow $p$-subgroup of $G$.
        \item The number of Sylow $p$-subgroups of $G$ is congruent to $1$ modulo $p$ and divides $\size{G}$.
        \item All Sylow $p$-subgroups of $G$ are conjugate.
    \end{enumerate}
\end{theorem}

\begin{proof}
    If $H \subgroup G$ with $\idx{G}{H}$ coprime with $p$, then by induction $H$ and therefore $G$ contains a Sylow $p$-group. Otherwise, by the class equation,
    $$\size{G} = \size{\centerizer(G)} +  \sum_{x} \idx{G}{N_x(G)}, $$
    it follows $\centerizer(G)$ is divisible by $p$ and thus $\gen{g} \subgroup \centerizer(G)$ for some $g \in \centerizer(G)$ with exponenet = $p$. Inducting on the order of $G$, $G/\gen{g}$ contains a Sylow $p$-subgroup, say $S/\gen{g}$ that is the image of $S \subgroup G$ that is a Sylow $p$-subgroup of $G$. 
    
    Let $P, Q \in \syl_p(G)$. $P$ does not normalize $Q$ because otherwise $PQ \subgroup G$ and $p^m = \size{PQ} > \size{P}$, a contradiction. Let $S = \set{P_1, \dots, P_k}$ be the conjugates of $P$ and let $\orbit{i}$ be the orbit of $P_i$ by the action $P$ on the  set $S$ by conjugation. Then $\size{\orbit{i}} = \idx{P}{N_P(P_i)} = \idx{P}{N_G(P_i) \cap P} = \idx{P}{P_i \cap P} \implies k = 1 \mod p.$

    If $P, Q \in \syl_p(G)$ are not conjugates, then $Q$ is not conjugate with conjugates of $P$.
    Consider the action of the elements of $Q$ on the set $\set{gPg^{-1} : g \in G} = \set{P_1, \dots, P_m}$. Then 
    $$\size{\orbit{P_i}} = \idx{Q}{N_Q(P_i)} = \idx{Q}{P_i \cap Q}, $$
    where the latter equality follows because $P_i(N_G(P_i) \cap Q)$ is a $p$-group that contains $P_i$ with order $\leq \size{P_i}$(a Sylow $p$-group) and thus $ N_G(P_i)\cap Q \leq P_i$. Since $Q$ is not a conjugate of $P$, $\idx{Q}{Q \cap P_i} = p^k, k > 0$ and $ \orbit{P_i}$ is divisible by $p$ and the number of conjugates of $P$ which is $\sum_i \size{\orbit{P_i}} = 0 \mod p$, a contradiction.
    \end{proof}


    \begin{theorem}
        If $\size{G}= pq$ for primes $p < q$, then $G = \ZZ/pq\ZZ$ if $p \nmid q - 1$ else $G = \C{pq}$ of $G =  \ZZ/q\ZZ \rtimes \C{p} $ for some non-trvial semi-direct product. 
    \end{theorem}

    \begin{proof}
        If $q > p$, $n_q = 1$ and thus $Q \in \syl_q(G)$ is normal. $|\aut(\C{q})| = q - 1$, therefore, there is a nontrivial map $\phi: \C{p}\to \aut(\C{q})$ if $p \mid q - 1$  
    \end{proof}
    \begin{theorem}[Fundamental Theorem of Finitely Generated Abelian Groups]

        Let $A$ be a finite abelian group and let $A(p)$ be the subgroup of all elements with order that is a power of $p$. Then 
        $$\prod_{A(p) \neq \set{1}} A(p) = A.$$
    
    \end{theorem}

    \begin{proof}
        Clearly the map $\phi: \prod_p A(p) \to A$ defined by $\phi((x_p)) = \prod_p x_p $ is an endomorphism.  We show that $\phi$ is injective and surjective. Let $\phi((x_p)) = 1$ for some $x = (x_p) \in \prod_p A(p)$. Let $q$ be a prime with $A(q) \neq \{1\}$. Then 
        $$x_q = \prod_{ p \neq q} x_p^{-1}.$$
        Let $m$ be the least common multiple of the primes powers  on the right hand side, i.e. powers of $p \neq q$. Then $x^{m}_q = 1$. But, $x_q^{q^r} = 1$ too. Consequently, $x_q^{(m, q^r)} = x_q^{1} = x_q = 1$. Thus $\prod_p x_p = 1$ iff all $x_p = 1$ and $\ker \phi = \set{1}.$

        To prove surjectivity, let $x \in A$ with $x^m = 1$ such that $m = \prod p_i^{r_i}$. By Euclidean algorithm,  $1= \sum_i u_i\prod_{j\neq i} p_j^{r_j}$ and thus $x = \prod_i x^{u_i\prod_{j\neq i} p_j^{r_j}}$ with 
        $ x^{u_i \prod_{j\neq i} p_j^{r_j}} \in A(p_i).$
    \end{proof}
    
   
    \textbf{Why nilpotence and the existence of normal Sylow sub-groups are equivalent?}: If $P, Q \in \syl_p(G)$ then $N_P(Q) = P\cap Q < P, Q$ and thus $Z(G)$ is always $ < G$. Thus $P = Q \iff G$ nilpotent.  

    \textbf{The number of ways $G$ acts on $H$}: $= \#$ of homomorphisms from $G$ to $\aut(H) = \#$ subgroups of order $|G|/|H^*|.$ 

    \begin{theorem}
        If $n \geq 5$ then $S_n$ is not solvable.
    \end{theorem}

    \begin{proof}
        Let $S_n$ decompose as $S_n = H_m \supset \cdots \supset H_0 = \set{1}$. Clearly, $S_n$ contains all $3$-cycles. We also know since $H_n/H_{n-1}$ is abelian $(abc)(ade)(acb)(aed) = (adebc)(aedcb) = (abd) \in H_{m-1}$. By induction all 3 cycles are in $\{1\}$, a contradiction.
    \end{proof}

    \begin{theorem}
        $A_n$ is simple for all $n \geq 5$.
    \end{theorem}
    \noindent \textit{A priori}: $A_n$ can be generated by $3$-cycles and $3$-cycles are conjugates.
    \begin{proof}
        Let $N \normal A_n.$ Let $\sigma \in N$. We show that $\sigma$ is a $3$-cycle or $\sigma = \id.$ The former implies $N = A_n$ and the latter implies $N$ is the trivial subgroup.  Let $\sigma$ have the maximal number of fixed points in $N$.

      Lrt all $\sigma$'s orbits have size $2$ and it does not fix elements $i, j$. If $\sigma$ is $(ijk)$ for some $k$, we are done. Otherwise,  $\gen{\sigma} > \gen{(ij)(rs)}$ for some $r, s$ because $\sigma$ is an even permutation and not a $3$-cycle. Let $\tau = (rsk)$ for some $k$. Then $\tau' = \tau \sigma \tau^{-1} \sigma^{-1}  \in N$. But $\tau' = (i, j)\sigma$ contradicting $\sigma$ fixes the maximal number of points. Thus at least one $\sigma$'s orbit has more than $2$ elements.

    Therefore, $\sigma = (ijk)(rs) \theta$ where $\theta$ is possible identity permutation. By similar argumenta as above picking $\tau' = (rsk)$, $\sigma$ can not be the element of $N$ with maximal fixed points unless it contains all of $A_n$.  
    \end{proof} 

    \subsection*{Properties of Common Non-Abelian Groups}
    \begin{itemize}
        \item \textit{Dihedral Group}: $\dihedral{2n}$
        \begin{itemize}
            \item  $\iso \C{n} \rtimes \C2$ acting by inversion
            \item $=\set{a, b | a^n, b^2, baba}$
        \end{itemize}

        \item \textit{Binary Dihedral Group/ Dicyclic Group}: $\dicyc(4n)$
        \begin{itemize}
            \item  $\iso \C{n} \rtimes \C4$ acting by inversion
            \item $=\set{a, b | a^n, b^4, baba}$
        \end{itemize}

        \item \textit{Generalized Quaternions}: $Q_{2^{n+2}}$
        \begin{itemize}
            \item  $\iso \C{2^n} \rtimes \C4$ acting by inversion
            \item $=\set{a, b | a^{2^n}, b^4, baba}$
        \end{itemize}

        \item \textit{Holomorph Group}: $\hol(G)$
        \begin{itemize}
            \item  $\iso G \rtimes \aut(G)$ 
            \item if $G$ is $\C{p}$, $p$ prime, $\hol(G)$ is isomorphic to the \textit{generalized affine group} 
        \end{itemize}
    \end{itemize}

    \subsection*{Notes on Category Theory}
    \begin{itemize}
        \item A category $\cat{C}$ is a collection of \textbf{objects} $\Ob(\cat{C})$, along with a set of maps, called \textbf{morphisms} between any two objects $A,B \in \Ob(\Cat)$ denoted by $\Mor(A, B)$.
        \item Morphisms follow the law of composition.
        \item Three axioms
        \begin{enumerate}
            \item \textbf{CAT 1} $\Mor(A, B)$ and $\Mor(A', B')$ are disjoint unless $(A, B) = (A', B')$, in which case they are equal.
            \item \textbf{CAT 2} For every $A \in \Ob(\Cat)$, there exists a morphism, $\id_A$ in $\Mor(A, A)$ that acts as a left and right identity for the elements of $\Mor(A, B)$ and $\Mor(B, A)$ resp. for all $B$.
            \item \textbf{CAT 3} The law of composition of morphisms is associative.
        \end{enumerate}
        \item The \textbf{operation} of a group $G$ on an object $A \in \Ob(\Cat)$ is a homomorphism from $G$ to $\aut(A).$ It is also called a \textbf{representation.}
        \item Given a category $\Cat$, we can construct a new category $\cat{D}$ where the objects are the morphisms of $\Cat$ and the morphisms between two objects $f, f'$ are  defined by a pair of momorphism $(\phi, \psi)$  that make the following diagram commute:
 
\[\begin{tikzcd}
	A && B \\
	{A'} && {B'}
	\arrow["f", from=1-1, to=1-3]
	\arrow["\phi"', from=1-1, to=2-1]
	\arrow["\psi", from=1-3, to=2-3]
	\arrow["{f'}"', from=2-1, to=2-3]
\end{tikzcd}\]

\label{universal_object}\item An object $P$ of a category $\Cat$ is called \textbf{universally attracting} (resp. \textbf{universally repelling}) if there is exists a \textit{unique} morphism from (resp. to) every object to(resp. from) $P$. If it is both, it is called \textbf{universal object}.

\end{itemize}    