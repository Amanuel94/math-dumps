\message{ !name(dedekind-rings.tex)}\documentclass{article}

\usepackage{amsmath}
\usepackage{amssymb}
\usepackage{amsthm}
\usepackage{hyperref}
\usepackage{bookmark}
% \usepackage{fbb}
% \usepackage{newpx}
% \usepackage{libertine}
% \usepackage{crimson}
% \usepackage{bm}
\usepackage{cochineal}

% \usepackage{eulervm} % math
\usepackage{mathpazo} % math

\usepackage{tikz-cd}
\usepackage{enumerate}

\newtheorem{theorem}{Theorem}
\newtheorem{lemma}[theorem]{Lemma}
\newtheorem{corollary}[theorem]{Corollary}
\newtheorem{proposition}[theorem]{Proposition}
\theoremstyle{definition}
\newtheorem{definition}[theorem]{Definition}
\newtheorem{example}[theorem]{Example}
\theoremstyle{remark}
\newtheorem{remark}[theorem]{Remark}

\hypersetup{
    colorlinks=true,
    linkcolor=blue,
    urlcolor=cyan,
    citecolor=red
}

\begin{document}

\message{ !name(dedekind-rings.tex) !offset(61) }
 shall be empty.

        \textit{Uniqueness}: Let $\ia = \ip_1 \cdots \ip_s = \iq_1 \cdots \iq_r$. We induct on $s$. Let $s = 1$. Then we have $\ip_1 = \iq_1 \cdots \iq_r$ for $r \geq 1$. Since all prime ideals are fintly generated, let $G$ be a set of generators of $\ip_1$. Since the product on the left is a subset of each $\q_i$, we have $\ip_1 \subseteq \iq_i$ for all $1 \leq i \leq r$. Take a generator $x_i \in \iq_i - G$ from each $\iq_i$. Then the product $x_1 \cdots x_r \in \ip_1$. By primality, one of $x_i \in \ip_1$, a contradiction. Thus  $\iq_i \subseteq \ip_1$ (and thus $\iq_i = \ip_1$) for some $i$, say $i = 1$. It follows that $\iq_2 \cdots \iq_r = \io$ and each $\iq_i = \io$ since prime ideals can not be inverses of each other.

        \newcommand{\inv}[1]{#1^{-1}}
        For the induction step, suppose the factorization is unique for all products up to $s - 1$ factors. We have $\ip_1 = \iq_1 \cdots \iq_{r}\inv{\ip_2} \cdots \inv{\ip_s}$.

\end{enumerate}

\end{document}

\message{ !name(dedekind-rings.tex) !offset(-52) }
