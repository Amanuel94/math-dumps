\documentclass{article}

\usepackage{amsmath}
\usepackage{amssymb}
\usepackage{amsthm}
\usepackage{hyperref}
\usepackage{bookmark}
\usepackage{fbb}
% \usepackage{newpx}
% \usepackage{libertine}
% \usepackage{crimson}
% \usepackage{bm}
% \usepackage{cochineal}

\usepackage{eulervm}
% \usepackage{mathpazo} % math

\usepackage{tikz-cd}
\usepackage{enumerate}

\newtheorem{theorem}{Theorem}
\newtheorem{lemma}[theorem]{Lemma}
\newtheorem{corollary}[theorem]{Corollary}
\newtheorem{proposition}[theorem]{Proposition}
\theoremstyle{definition}
\newtheorem{definition}[theorem]{Definition}
\newtheorem{example}[theorem]{Example}
\theoremstyle{remark}
\newtheorem{remark}[theorem]{Remark}

\hypersetup{
    colorlinks=true,
    linkcolor=blue,
    urlcolor=cyan,
    citecolor=red
}

\begin{document}

\title{Lang's Algebra Chapter 3 Solutions}
\author{Amanuel Tewodros}
% \maketitle

% \tableofcontents

% commands

\newcommand{\size}[1]{|#1|}
\newcommand{\idx}[2]{[#1 : #2]}
\newcommand{\gen}[1]{\langle #1 \rangle}
\newcommand{\set}[1]{\{#1\}}
\newcommand{\normal}{\trianglelefteq}
\newcommand{\subgroup}{\leq}
\newcommand{\normalizer}[1]{\text{N}_{#1}}
\newcommand{\centerizer}{Z}
\newcommand{\aut}{\text{Aut}}
\newcommand{\inn}{\text{Inn}}
\newcommand{\syl}{\text{Syl}}
\newcommand{\sym}{\text{Sym}}
\newcommand{\alt}{\text{Alt}}
\newcommand{\id}{\text{id}}
\newcommand{\im}{\text{im}}
\newcommand{\orbit}[1]{\mathcal{O}_{#1}}
\newcommand{\ZZ}{\mathbb{Z}}
\newcommand{\C}[1]{\mathbb{Z}/#1\mathbb{Z}}
\newcommand{\iso}{\cong}
\newcommand{\dihedral}[1]{D_{#1}}
\newcommand{\dicyc}{\text{DiC}}
\newcommand{\hol}{\text{Hol}}
\newcommand{\I}[1]{\mathfrak{#1}} 
\newcommand{\Map}{\mathrm{Map}}
\newcommand{\nequiv}{\not \equiv}
\newcommand{\cat}[1]{\mathcal{#1}}
\newcommand{\Cat}{\mathcal{C}}
\newcommand{\Mor}{\text{Mor}}
\newcommand{\Ob}{\text{Ob}}
\newcommand{\End}{\text{End}}
\newcommand{\horline}{\noindent\rule{\textwidth}{1pt} \newline}
\newcommand{\Img}{\mathrm{Im}\ }
\newcommand{\Hom}{\mathrm{Hom}}
\newcommand{\fto}[1]{\overset{#1}{\to}}
\newcommand{\from}{\leftarrow}
\newcommand{\m}[1]{\mathfrak{#1}}
\newcommand{\dual}[1]{#1^{\vee}}
\newcommand{\ddual}[1]{#1^{\vee \vee}}
\newcommand{\rank}{\mathrm{rank}}
\newcommand{\tor}[1]{#1_{\mathrm{tor}}}
\newcommand{\dirlim}{\underrightarrow{\lim}}
\newcommand{\localize}[2]{{#1}^{-1}#2}
% \newcommand{\rank}[1]{def}
This piece contains the solutions for Serg Lang's Graduate Algebra Chapter II exercises on Dedekind rings.
\begin{enumerate}
        \newcommand{\io}{\I o}
        \newcommand{\ia}{\I a}
        \newcommand{\ib}{\I b}
        \newcommand{\ic}{\I c}
        \newcommand{\ip}{\I p}
        \newcommand{\iq}{\I q}
  \item[(13)] Since the ideals of $\io$ are fractional ideals by definition, for a given ideal $\ia$ of $\io$, there is a fractional ideal $\ib \subset K$ with $c\ib \subset \io$ such that $\ia \ib = \io.$ Since, the unit $1_K$ is a quotient of $\io$, $1_K \in \io$. Therefore, let $1_K = \sum_{i \leq n} a_ib_i$ where $a_i \in \ia$ and $b_i \in \ib.$ For any element $a \in \ia$, then we have $a = \sum_{i} aa_ib_i$. But $ca = \sum_i a_ib_i'$ where $b_i' = acb_i \in \io$. If $a \neq b$ then $ca \neq cb$, thus $\ia$ is generated by $\set{a_1, \dots, a_n}.$

  \item[(14)]  \textit{Existence}: Let $S$ be the set of all ideals of $\io$ that don't have prime factorization. Suppose $S$ is not empty and let $\ia_1  \in S.$ Then consider the asending chain of ideals in $S$
        $$\ia_1 \subseteqq \ia_2 \subseteqq ... $$
        Since, the union ideal $\ia =  \cup_i \ia_i$ is an ideal in $\io$, it is fintely generated and  $\ia = \ia_n$ for some $n$. It follows the the chain is finite and any $\ib \supset \ia$ admits prime factorization. For all $xy \in \ia$ and if either $x$ or $y$ is an element of $\ia$, then $\ia$ is a prime and there is nothing left to prove. Otherwise, let $\ia = \gen{a_1, \dots, a_m}$. Then $\I s = \gen{a_1, \dots, a_m, x} \supset \ia$, $\I t = \gen{a_1, \dots, a_m, y} \supset \ia$, and we have $\I s\I t \subseteq \ia$ ( hence $\ia = \I s\I t$). Thus $\ia \notin S$ and by induction, $S$ shall be empty.

        \textit{Uniqueness}: Let $\ia = \ip_1 \cdots \ip_s = \iq_1 \cdots \iq_r$. We induct on $s$. Let $s = 1$. Then we have $\ip_1 = \iq_1 \cdots \iq_r$ for $r \geq 1$. Since all prime ideals are fintly generated, let $G$ be a set of generators of $\ip_1$. Since the product on the left is a subset of each $\iq_i$, we have $\ip_1 \subseteq \iq_i$ for all $1 \leq i \leq r$. Take a generator $x_i \in \iq_i - G$ from each $\iq_i$. Then the product $x_1 \cdots x_r \in \ip_1$. By primality, one of $x_i \in \ip_1$, a contradiction. Thus  $\iq_i \subseteq \ip_1$ (and thus $\iq_i = \ip_1$) for some $i$, say $i = 1$. It follows that $\iq_2 \cdots \iq_r = \io$ and each $\iq_i = \io$ since prime ideals can not be inverses of each other.

        \newcommand{\inv}[1]{#1^{-1}}
        For the induction step, suppose the factorization is unique for all products up to $s - 1$ factors. By similar reasoning as above, let $x_i \in \iq_i - G$ where $G$ is the generator of $\ip_1$. Then $x_1 \cdots x_r \in \prod \iq_i =  \prod \ip_i \subseteq \ip_1 \cap \cdots \cap \ip_s \subseteq \ip_1.$ By primality, one of $x_1 \in \ip_1$ contradicting the inexistence of $x_i$ in $G \subseteq \ip_1$. Thus $\ip_1 \supseteq \iq_j$ for some $j$. By maximality of prime ideals, $\ip_1 = \iq_j$. By cancellation and induction, the statement follows.
  \item[(15)] By unique factorization, we know $(t)=  \ip^s.$ We also have $\ip^{s} \subseteq \ip^{s-1} \cap \ip$ for  $s \geq 1.$ Thus $(t) = \ip.$

\newcommand{\ploc}[1]{#1_{\ip}}
  \item[(16)] First, we show that $\ploc{\io}$ is a Dedekind ring. Let $S = A - \ip$, and let $\I a$ and $\I b$ be ideals of $\io$. First, we note that if $\I a$ is a fractional ideal, so is $\localize{S}{\I a}$. If $xa \in \ia$ for all $x \in \io, a \in \ia$, then $\frac{x}s \frac{a}t = \frac{xa}{st}$, which is an element of $\localize{S}{\I a}$ by multiplicativeness of $S$. Similarly if $c\ia \subseteq \io$ for some $c \in \io$, then $\frac{c}1 (\localize{S}{\I a}) \subseteq \ploc{\io}.$

        For elements $a_i \in \I a$, $b \in \I b$ and any elements $s_i, t_i \in S$, we have the finite sum $\sum \frac{a_i}{s_i}\frac{b_i}{t_i} = \sum \frac{a_ib_i}{s_it_i} = \frac{1}{x}\sum{a_i'b_i}$ where $a'_i = \prod_{j \neq i}s_jt_ja_i$  and $x = \prod_{i}s_it_i$. Therefore, $\localize{S}{\I a}\cdot \localize{S}{\I b} \subseteq \localize{S}{\I a\I b}$. For the reverse inclustion, let $r/s = \sum_i a_ib_i /s$, then picking $a_i' = a_i/s$ and $b_i' = bi/1$, we have $r/s = \sum_i a_i'b_i'$. This proves that localization by $\ip$ is multiplicative.

        The group properties of the set of fractional ideals of $\ploc{\io}$ then directly follows from the group properties of that of $\io$. It remains to show that there is one prime ideal in $\ploc{\io}.$ By multiplicativeness of the homomorphism $\ia \mapsto \localize{S}{\ia}$, and the unique factorization proved in the previous excercise, we can express any ideal $\I s$ of $\ploc{\io}$ as
        $$\I s = \localize{S}{\I q_1} \cdots \localize{S}{\I q_m.}$$
        At most one of $\localize{S}{\iq_i}$ is equal to $\ip$ (up to uniqueness) and the rest are units. Thus the only prime ideal is $\localize{S}\ip.$

  \item[(17)]
        \begin{enumerate}[(a)]
          \item If $\I a \mid \I b$, by definition there is an ideal $\I c$ such that $\I b  = \I a \I c  \subseteq \ia  \io = \I a.$ On the other hand,  $\I \I a^{-1}\I b \subseteq a^{-1}\I a = \io$. From the definiton of the fractional ideals, it follows that $\ia^{-1}\ib$ is an ideal of $\ia$. The backward direction follows immediately.


          \item For  ideals $\ia, \ib$ and $\ic$, $\ic\ia + \ic\ib$ is the set of all finite sums $\sum_i c_ia_i + \sum_j c_jb_j$ where $a_i \in \ia$, $b_j \in \ib$ and $c_i, c_j \in \ic$. By rearranging the terms, we can write this sum as $\sum_i c_i(a_i + b_i)+ \sum_j c_j(a_j + 0)+ \sum_k c_k(0 + b_k)$. Hence $\ic(\ia + \ib) \supseteq \ic\ia + \ic\ib.$ The reverse inclustion follows from the distributive property of $(+)$ over $(\cdot)$. Therefore, $\ic(\ia + \ib) = \ic\ia + \ic\ib$.

                Now, let $\I d \mid \ia$ and $\I d \mid \ib$. Then we have $\ia + \ib = \I d\ia' + \I d\ib' = \I d(\ia' + \ib')$ for some ideals $\ia', \ib'$. Thus $\I d$ also divides $\ia + \ib.$
        \end{enumerate}

  \item[(18)] Suppose $\ip \subsetneq \ia \subsetneq \io$. Then by the above exercise, $\ia \mid \ip$, i.e. $\ip = \ia\ic$. But since $\ip \neq \ia$, $\ic \neq \io$ and distinct factorizations of $\ip$ exist, a contradiction.

  \item[(19)] Let $\ic = \ia + \ib$ and let $\prod_i\ip_i^{r_i}$ be the factorization of $\ia$. By the previous problem we can find $x \in \ia$ such that
        $$x =
        \begin{cases}
          0 \mod \ip_i^{r_i} \text{ if } \ip^{r_i}_i \mid \ic \\
          1 \mod \ip^{r_i}_i \text{ otherwise }
        \end{cases}
        $$

    Now $x^{-1}\ia \subseteq \ia^{-1}\ia  = \io$. But

\end{enumerate}

\end{document}
