\documentclass{article}

\usepackage{amsmath}
\usepackage{amssymb}
\usepackage{amsthm}
\usepackage{hyperref}
\usepackage{bookmark}
% \usepackage{fbb}
% \usepackage{newpx}
% \usepackage{libertine}
% \usepackage{crimson}
% \usepackage{bm}
\usepackage{cochineal}

% \usepackage{eulervm} % math
\usepackage{mathpazo} % math

\usepackage{tikz-cd}
\usepackage{enumerate}

\newtheorem{theorem}{Theorem}
\newtheorem{lemma}[theorem]{Lemma}
\newtheorem{corollary}[theorem]{Corollary}
\newtheorem{proposition}[theorem]{Proposition}
\theoremstyle{definition}
\newtheorem{definition}[theorem]{Definition}
\newtheorem{example}[theorem]{Example}
\theoremstyle{remark}
\newtheorem{remark}[theorem]{Remark}

\hypersetup{
    colorlinks=true,
    linkcolor=blue,
    urlcolor=cyan,
    citecolor=red
}

\begin{document}

% \title{Notes on Serge Lang's Algebra}
% \author{Amanuel Tewodros}


% commands

\newcommand{\size}[1]{|#1|}
\newcommand{\idx}[2]{[#1 : #2]}
\newcommand{\gen}[1]{\langle #1 \rangle}
\newcommand{\set}[1]{\{#1\}}
\newcommand{\normal}{\trianglelefteq}
\newcommand{\subgroup}{\leq}
\newcommand{\normalizer}[1]{\text{N}_{#1}}
\newcommand{\centerizer}{Z}
\newcommand{\aut}{\text{Aut}}
\newcommand{\inn}{\text{Inn}}
\newcommand{\syl}{\text{Syl}}
\newcommand{\sym}{\text{Sym}}
\newcommand{\alt}{\text{Alt}}
\newcommand{\id}{\text{id}}
\newcommand{\im}{\text{im}}
\newcommand{\orbit}[1]{\mathcal{O}_{#1}}
\newcommand{\ZZ}{\mathbb{Z}}
\newcommand{\C}[1]{\mathbb{Z}/#1\mathbb{Z}}
\newcommand{\iso}{\cong}
\newcommand{\dihedral}[1]{D_{#1}}
\newcommand{\dicyc}{\text{DiC}}
\newcommand{\hol}{\text{Hol}}
\newcommand{\I}[1]{\mathfrak{#1}} 
\newcommand{\Map}{\mathrm{Map}}
\newcommand{\nequiv}{\not \equiv}
\newcommand{\cat}[1]{\mathcal{#1}}
\newcommand{\Cat}{\mathcal{C}}
\newcommand{\Mor}{\text{Mor}}
\newcommand{\Ob}{\text{Ob}}
\newcommand{\End}{\text{End}}
\newcommand{\horline}{\noindent\rule{\textwidth}{1pt} \newline}
\newcommand{\Img}{\mathrm{Im}\ }
\newcommand{\Hom}{\mathrm{Hom}}
\newcommand{\fto}[1]{\overset{#1}{\to}}
\newcommand{\from}{\leftarrow}
\newcommand{\m}[1]{\mathfrak{#1}}
\newcommand{\dual}[1]{#1^{\vee}}
\newcommand{\ddual}[1]{#1^{\vee \vee}}
\newcommand{\rank}{\mathrm{rank}}
\newcommand{\tor}[1]{#1_{\mathrm{tor}}}
\newcommand{\dirlim}{\underrightarrow{\lim}}

The next four exercises develop the concept of \textit{direct limits} and the ``dual'' notion of \textit{inverse limits}. In these exercises $I$ is a nonempty index set with a partial order $\le$ (cf. Appendix I). For each $i \in I$, let $A_i$ be an additive abelian group. In Exercise 8 assume also that $I$ is a \textit{directed set}: for every $i, j \in I$ there is some $k \in I$ with $i \le k$ and $j \le k$.

\bigskip

\noindent \textbf{8.} Suppose for every pair of indices $i, j$ with $i \le j$ there is a map $\rho_{ij} : A_i \to A_j$ such that the following hold:
\begin{enumerate}
    \item[(i)] $\rho_{ik} = \rho_{jk} \circ \rho_{ij}$ whenever $i \le j \le k$, and
    \item[(ii)] $\rho_{ii} = 1$ for all $i \in I$.
\end{enumerate}

Let $B$ be the disjoint union of all the $A_i$. Define a relation $\sim$ on $B$ by
\[
a \sim b \quad \text{if and only if there exists } k \text{ with } i, j \le k \text{ and } \rho_{ik}(a) = \rho_{jk}(b),
\]
for $a \in A_i$ and $b \in A_j$.

\begin{enumerate}
    \item[(a)] Show that $\sim$ is an equivalence relation on $B$. (The set of equivalence classes is called the \textit{direct} or \textit{inductive limit} of the directed system $\{A_i\}$, and is denoted $\varinjlim A_i$. In the remaining parts of this exercise let $A = \varinjlim A_i$.)
    
    \item[(b)] Let $\bar{x}$ denote the class of $x$ in $A$ and define $\rho_i : A_i \to A$ by $\rho_i(a) = \bar{a}$. Show that if each $\rho_{ij}$ is injective, then so is $\rho_i$ for all $i$ (so we may then identify each $A_i$ as a subset of $A$).
    
    \item[(c)] Assume all $\rho_{ij}$ are group homomorphisms. For $a \in A_i$, $b \in A_j$ show that the operation
    \[
    \bar{a} + \bar{b} =\overline{\rho_{ik}(a) + \rho_{jk}(b)}
    \]
    where $k$ is any index with $i, j \le k$, is well defined and makes $A$ into an abelian group. Deduce that the maps $\rho_i$ in (b) are group homomorphisms from $A_i$ to $A$.
    
    \item[(d)] Show that if all $A_i$ are commutative rings with $1$ and all $\rho_{ij}$ are ring homomorphisms that send $1$ to $1$, then $A$ may likewise be given the structure of a commutative ring with $1$ such that all $\rho_i$ are ring homomorphisms.
    
    \item[(e)] Under the hypotheses in (c) prove that the direct limit has the following \textit{universal property}: if $C$ is any abelian group such that for each $i \in I$ there is a homomorphism $\varphi_i : A_i \to C$ with $\varphi_i = \varphi_j \circ \rho_{ij}$ whenever $i \le j$, then there is a unique homomorphism $\varphi : A \to C$ such that $\varphi \circ \rho_i = \varphi_i$ for all $i$.
\end{enumerate}

\textit{ Sol. }
\begin{enumerate}[(a)]
    \item Let $x \in B$. Then there is $s$ such that $x \in A_s$. Choosing $i = j = k = s$, we  see that $\sim$ is \textit{relfexive}.  By symmetry  of $=$, the symmetry of $\sim$ follows directly. Let $a \sim b$ and $b \sim c$. Let $\rho_{ik}(a) = \rho_{jk}(b)$ and let $\rho_{jt}(b) = \rho_{st}(c)$. WLOG, let $k \leq t$. Then $\rho_{it}(a) = \rho_{kt}\circ\rho_{ik}(a) = \rho_{kt}\circ \rho_{jk}(b) = \rho_{jt}(b) = \rho_{st}(c).$ Thus $\sim$ is transitive.
    
    \item Let $a, b \in A_i$ with $a \neq b.$ By injectivity, $\rho_{ik}(b) \neq \rho_{ik}(b)$ for all $k \geq i, j$. Thus, $a \nsim b$.
    
    \item For the addition to be well-defined, it should have the same value regardless of the choice of $a$ and $b$ as long as  they are picked for their respective equivalence classes. Let $x \sim a$ and $y \sim b.$ Let $\rho_{it}(a) = \rho_{st}(x)$ and $\rho_{je}(b) = \rho_{de}(y).$ WLOG, let $t \geq e.$ If $k \geq t$, we are done. Otherwise,  $\rho_{kt}(\rho_{ik}(a) + \rho_{jk}(b)) = \rho_{it}(a) + \rho_{jt}(b) = \rho_{st}(x) + \rho_{dt}(y) = \rho_{et}(\rho_{se}(x) + \rho_{de}(y)).$ Thus $+$ is well-defined.
    
    $A$ is then an abelian group because if $\bar{a}, \bar{b} \in A$, then $\bar{a} -  \bar{b} \in A$ and $\bar{0} \supseteq \set{0_{A_i}}_{i \in I} \in A.$ It follows that $\rho_i$ are group homomorphisms because  $\rho_i(a + b) = \overline{a + b} = \bar{a} + \bar{b}$ (taking $k = i$) $ = \rho_i(a) + \rho_i(b).$

    \item $A$ is still an additive abelian group but now commutitive multiplicative structure is built upon it. The multiplication given by 
        $$\bar{a} \cdot \bar{b} = \overline{\rho_{ik}(a) \cdot \rho_{jk}(b)}$$
    for all $k \geq i, j$ is well defined and the proof is similar to the one given in (c) as $\rho_{ij}$ are ring homomorphisms.  Furthermore, $\bar{a}\cdot(\bar{b} + \bar{c}) = \overline{\rho_{ik}(a) \cdot (\rho_{mk}(b) + \rho_{nk}(c)})$ for $k \geq i, m,n$. The distributive property of $(\cdot)$ in $A$ follows from the distributive property $(\cdot)$ in $A_i$ once we note that $a \sim \rho_{ik}(a)$ for all $k \geq  i$.

    \item We define $\varphi: A \to C$ as follows,
    
        $$\varphi(\bar{x}) = \varphi_i(x),\quad x \in A_i.$$
    
    We first show that this definition is independent of the choice of the representative $x$. Let $x \sim y$, i.e., $\rho_{ik}(x) = \rho_{jk}(y)$. 

    \begin{eqnarray*}
        \varphi(\bar{x}) & = & \varphi_i(x)\\
        & = & \varphi_k(\rho_{ik}(x)) \\
        & = & \varphi_k(\rho_{jk}(y))\\
        & = & \varphi_j(y).
    \end{eqnarray*}

    Thus, $\varphi$ is well defined. Since $A$ is a disjoint union of $A_i$ modulo $\sim$, $\varphi$ is defined everywhere in $A$ and uniquness follows from definition. \qed
\end{enumerate} 
\noindent
\textbf{9.} Let $I$ be the collection of open intervals $U = (a, b)$ on the real line containing a fixed real number $p$. Order these by reverse inclusion: $U \leq V$ if $V \subseteq U$ (note that $I$ is a directed set). For each $U$ let $A_U$ be the ring of continuous real valued functions on $U$. For $V \subseteq U$ define the \emph{restriction maps} 
\[
\rho_{UV} : A_U \to A_V \quad \text{by} \quad f \mapsto f|_V,
\]
the usual restriction of a function on $U$ to a function on the subset $V$ (which is easily seen to be a ring homomorphism). Let
\[
A = \varinjlim A_U
\]
be the direct limit. In the notation of the preceding exercise, show that the maps $\rho_U : A_U \to A$ are \emph{not} injective but are all surjective ($A$ is called the ring of \emph{germs of continuous functions at} $p$).

\textit{ Sol. } First, we need to describe $A$. $A$ consists of equivalence classes that contain real-valued continuous functions that agree on some open interval containing $p$. That is, $f \sim g$ iff $f|_X = g|_X$ for some open interval $X$ containg $p$.

To show that $\rho_U$ is not injective, consider an interval $U = (a, b)$ and let $X = (a, \frac{b + p}{2})$. Define the funtions $f, g \in A_U$ as follows

$$
f(x) = \begin{cases}
    0 \quad \quad \quad \quad \text{ if } x \in X \\
    x - \frac{b + p}2 \quad \text{ otherwise }
\end{cases}
\quad 
\text{ and }
\quad
g(x) = \begin{cases}
    0 \quad \quad \quad \quad \quad \ \  \text{ if } x \in X \\
    2\big(x - \frac{b + p}2\big) \quad \text{ otherwise }
\end{cases}
$$
Both $f$ and $g$ are continuous and $f \sim g$ because $f|_X = g|_X.$ Thus $\rho_U$ is not injective (alternatively, one can show the kernel of the this map contains all elements of $A_U$ that agree with the zero function in some open interval $X$ containing $p$).

Next, we show surjectivity. Let $F \in A$. Since $F$ is an equivalence class, pick some $f: (a, b) \to \mathbb R \in F$. Let $U = (a', b')$. Then define the $f': U \to  \mathbb{R}$ as 
$$
f'(x) = \begin{cases}
    f(x) \quad \text{ if } x \in U \cap \big(\frac{a+p}{2}, \frac{b+p}{2}\big) \\
    f\big(\frac{a + p}{2}\big) \quad \text{ if } x \in U \cap (-\infty, (a + p)/2] \\
    f\big(\frac{b + p}2\big) \quad \text{ if } x \in U \cap [(b+p)/2, \infty) 
\end{cases}$$
It follows that $f'\in A_U$ and $\rho_U(f') = F.$  \qed

We now develop the notion of inverse limits. Continue to assume $I$ is a partially ordered set (but not necessarily directed), and $A$ is a group for all $i \in I$.

\noindent
\textbf{10.} Suppose for every pair of indices $i, j$ with $i \leq j$ there is a map $\mu_{ji} : A_j \to A_i$ such that the following hold:
\begin{enumerate}
    \item[(i)] $\mu_{ji} \circ \mu_{kj} = \mu_{ki}$ whenever $i \leq j \leq k$, and
    \item[(ii)] $\mu_{ii} = 1$ for all $i \in I$.
\end{enumerate}
Let $P$ be the subset of elements $(a_i)_{i \in I}$ in the direct product $\prod_{i \in I} A_i$ such that $\mu_{ji}(a_j) = a_i$ whenever $i \leq j$ (here $a_i$ and $a_j$ are the $i^{\text{th}}$ and $j^{\text{th}}$ components respectively of the element in the direct product). The set $P$ is called the \textit{inverse} or \textit{projective limit} of the system $\{A_i\}$, and is denoted $\varprojlim A_i$.

\begin{enumerate}
    \item[(a)] Assume all $\mu_{ji}$ are group homomorphisms. Show that $P$ is a subgroup of the direct product group (cf. Exercise 15, Section 5.1).
    
    \item[(b)] Assume the hypotheses in (a), and let $I = \mathbb{Z}^+$ (usual ordering). For each $i \in I$ let $\mu_i : P \to A_i$ be the projection of $P$ onto its $i^{\text{th}}$ component. Show that if each $\mu_{ji}$ is surjective, then so is $\mu_i$ for all $i$ (so each $A_i$ is a quotient group of $P$).
    
    \item[(c)] Show that if all $A_i$ are commutative rings with $1$ and all $\mu_{ji}$ are ring homomorphisms that send $1$ to $1$, then $A$ may likewise be given the structure of a commutative ring with $1$ such that all $\mu_i$ are ring homomorphisms.
    
    \item[(d)] Under the hypotheses in (a) prove that the inverse limit has the following \textit{universal property}: If $D$ is any group such that for each $i \in I$ there is a homomorphism $\pi_i : D \to A_i$ with $\pi_i = \mu_{ji} \circ \pi_j$ whenever $i \leq j$, then there is a unique homomorphism $\pi : D \to P$ such that $\mu_i \circ \pi = \pi_i$ for all $i$.
\end{enumerate}

\textit{ Sol. }
\begin{enumerate}[(a)]
    \item Since $P \subseteq \prod_i A_i$, it suffieces to prove that $P$ is a group. Let $(a_i) \in P$. Since $(0, \dots) \in P$ and $(0, \dots) + (a_i)_{i\in I} = (a_i + 0)$, $P$ contains the identity. Let 
    $(a_i), (b_i) \in P$. Then $(a_i) - (b_i) = (a_i - b_i) \in P$ because $\mu_{ij}$ are additive.

    \item If $\mu_{ij}$ is surjective, for every element $a_i \in A_i$ there is an element $a_{i+1} \in A_{i+1}$ such that $\mu_{i+1, i}(a_{i+1}) = a_i$. Inducting on $i$, it follows that for every $a_i \in A_i$, there is an element $a \in P$ such that $a$ has $a_i$ at the $i$-th component. It immediately follows the projection $\mu_i$ is onto the $A_i$.
    
    \item If $(*)$ is a commutative binary operator such that  $\mu_{ij}$ is linear in $(*)$, then for two elements $a, b \in P$ $a*b \in P$ and is well defined if $(*)$ is well defined in $A_i$. To show that consider the expression between $a = (a_i)$ and $b = (b_i)$. If we define $a*b$ as $(a_i * b_i)_{i \in I}$. Since, by assumption,  $\mu_{ij}$ linear in $(*)$, this product is well-defined. This shows that $(+)$ and $(\cdot)$ are well-defined in $A$. (For multiplication, the assumption that $\mu_{ij}(1) = 1$ is important to ensure the consisency of the relation $1 + 1 = 2 \cdot 1$ in $A_i$). Distributive property follows immediately.
    
    \item We show that $\pi: D \to A$ defined $\pi(d) = (\pi_i(d))_{i \in I}$ satisfies the universal property. The $\pi$ is clearly homomorphism because each $\pi_i$ is homomorphism and addtion (and multiplication) are defined component-wise. Let $\pi_i = \mu_i \circ \pi = \mu_i \circ \pi'.$ It follows $0 = \mu_i \circ(\pi - \pi')$. This can only happen $\pi  = \pi'$, hence uniquness. \qed
\end{enumerate}


\noindent \textbf{11.} Let $p$ be a prime let $I=\mathbb{Z}^{+}$, let $A_{i}=\mathbb{Z}/p^{i}\mathbb{Z}$ and let $\mu_{ji}$ be the natural projection maps

\[
\mu_{ji}:a\ (\mathrm{mod}\ p^{j})\longmapsto a(\mathrm{mod}\ p^{i}).
\]

The inverse limit $\varprojlim\mathbb{Z}/p^{i}\mathbb{Z}$ is called the ring of \textit{$p$-adic integers}, and is denoted by $\mathbb{Z}_{p}$.

\begin{enumerate}[(a)]
    \item Show that every element of $\mathbb{Z}_{p}$ may be written uniquely as an infinite formal sum $b_{0}+b_{1}p+b_{2}p^{2}+b_{3}p^{3}+\cdots$ with each $b_{i}\in\{0,\,1,\ldots,\,p-1\}$. Describe the rules for adding and multiplying such formal sums corresponding to addition and multiplication in the ring $\mathbb{Z}_{p}$. [Write a least residue in each $\mathbb{Z}/p^{i}\mathbb{Z}$ in its base $p$ expansion and then describe the maps $\mu_{ji}$.] (Note in particular that $\mathbb{Z}_{p}$ is uncountable.)

    \item Prove that $\mathbb{Z}_{p}$ is an integral domain that contains a copy of the integers.

    \item Prove that $b_{0}+b_{1}p+b_{2}p^{2}+b_{3}p^{3}+\cdots$ as in (a) is a unit in $\mathbb{Z}_{p}$ if and only if $b_{0}\neq 0$.

    \item Prove that $p\mathbb{Z}_{p}$ is the unique maximal ideal of $\mathbb{Z}_{p}$ and $\mathbb{Z}_{p}/p\mathbb{Z}_{p}\cong\mathbb{Z}/p\mathbb{Z}$ (where $p=0+1p+0p^{2}+0p^{3}+\cdots$). Prove that every ideal of $\mathbb{Z}_{p}$ is of the form $p^{n}\mathbb{Z}_{p}$ for some integer $n\geq 0$.

    \item Show that if $a_{1}\not\equiv 0\pmod{p}$ then there is an element $a=(a_{i})$ in the direct limit $\mathbb{Z}_{p}$ satisfying $a_{j}^{P}\equiv 1\pmod{p^{j}}$ and $\mu_{j\,1}(a_{j})=a_{1}$ for all $j$. Deduce that $\mathbb{Z}_{p}$ contains $p-1$ distinct $(p-1)^{st}$ roots of $1$.
\end{enumerate}

\textit{ Sol. }
\begin{enumerate}
    \item 
\end{enumerate}

\end{document}