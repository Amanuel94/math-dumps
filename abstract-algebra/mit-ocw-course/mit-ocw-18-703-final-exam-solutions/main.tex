\documentclass[12pt]{article}
\usepackage{fullpage, enumitem, amsmath, amssymb, graphicx}

% Title and Author
\title{18-703 Modern Algebra: Spring 2013 Final Exam Answers}
\author{Amanuel Tewodros Getachew}

\begin{document}
\maketitle

\noindent
\rule{\linewidth}{0.4pt}

\section*{Problem 1}

\begin{enumerate}[label=\roman*., leftmargin=*, align=left]
    \item \textbf{Definition of a Group:}  
    A group is a set $G$ equipped with a binary operation $(\cdot): G \times G \to G$ satisfying the following properties:
    \begin{enumerate}[label=\alph*., leftmargin=*]
        \item \textit{Associativity:} For all $a, b, c \in G$, $(a \cdot b) \cdot c = a \cdot (b \cdot c)$.
        \item \textit{Identity Element:} There exists an element $e \in G$ such that $e \cdot a = a \cdot e = a$ for all $a \in G$.
        \item \textit{Inverse Element:} For every $a \in G$, there exists an element $b \in G$, denoted $a^{-1}$, such that $a \cdot b = b \cdot a = e$.
    \end{enumerate}

    \item \textbf{Definition of an Automorphism:}  
    An automorphism $\varphi$ of a group $G$ is a bijective homomorphism $\varphi: G \to G$, satisfying $\varphi(a \cdot b) = \varphi(a) \cdot \varphi(b)$ for all $a, b \in G$.

    \item \textbf{Definition of the Dihedral Group $D_n$:}  
    $D_n$ is the group of symmetries of a regular $n$-gon, including rotations and reflections. Formally,  
    \[
    D_n := \{r^i s^j : r^n = 1, s^2 = 1, sr = r^{-1}s \},
    \]
    where $r$ represents rotation and $s$ represents reflection.

    \item \textbf{Definition of an Ideal:}  
    An ideal $I$ is a subset of a ring $R$ such that for all $r \in R$, $rI \subseteq I$ and $Ir \subseteq I$.

    \item \textbf{Definition of a Principal Ideal Domain (PID):}  
    A principal ideal domain is an integral domain $R$ in which every ideal is generated by a single element. That is, for any ideal $I \triangleleft R$, there exists $a \in R$ such that 
    \[
    I = (a) = \{ra : r \in R\}.
    \]

    \item \textbf{Definition of a Unique Factorization Domain (UFD):}  
    A unique factorization domain is an integral domain $R$ in which every nonzero element $a \in R$ can be expressed uniquely (up to units) as a product of irreducibles. Formally, if $a = p_1 \cdots p_r = q_1 \cdots q_s$, where $p_i, q_i$ are irreducibles, then $r = s$ and $p_i = u_i q_i$ for some unit $u_i \in R$.
\end{enumerate}

\section*{Problem 2}

\begin{enumerate}[label=\roman*., leftmargin=*, align=left]
    \item \textbf{Equivalence Relation on a Group:}  
    Let $G$ be a group and define a relation $g_1 \sim g_2$ if there exists $h \in G$ such that $g_1 = hg_2h^{-1}$. Prove that $\sim$ is an equivalence relation:
    \begin{enumerate}[label=\alph*., leftmargin=*]
        \item \textit{Reflexive:} For any $a \in G$, $a = haa^{-1}$. Thus, $a \sim a$.
        \item \textit{Symmetric:} If $g_1 \sim g_2$, then $g_1 = hg_2h^{-1}$. Multiplying by $h^{-1}$ on both sides, $h^{-1}g_1h = g_2$. Thus, $g_2 \sim g_1$.
        \item \textit{Transitive:} If $g_1 \sim g_2$ and $g_2 \sim g_3$, then $g_1 = h_1g_2h_1^{-1}$ and $g_2 = h_2g_3h_2^{-1}$. Substituting, $g_1 = h_1(h_2g_3h_2^{-1})h_1^{-1} = (h_1h_2)g_3(h_1h_2)^{-1}$. Thus, $g_1 \sim g_3$.
    \end{enumerate}

    \item \textbf{Equivalence Classes in $S_5$:}  
    Let $\overline{\sigma}$ denote the equivalence class containing $\sigma \in S_5$. By the theorem on conjugacy in symmetric groups, two permutations $\sigma_1, \sigma_2 \in S_5$ are conjugate if and only if they have the same cycle type. Thus, the equivalence classes in $S_5$ are:
    \begin{enumerate}[label=(\arabic*), leftmargin=*]
        \item $\overline{1} = \{1\}$
        \item $\overline{(1\,2)} = \{\sigma : \text{type}(\sigma) = (2, 1, 1, 1)\}$
        \item $\overline{(1\,2)(3\,4)} = \{\sigma : \text{type}(\sigma) = (2, 2, 1)\}$
        \item $\overline{(1\,2\,3)} = \{\sigma : \text{type}(\sigma) = (3, 1, 1)\}$
        \item $\overline{(1\,2\,3)(4\,5)} = \{\sigma : \text{type}(\sigma) = (3, 2)\}$
        \item $\overline{(1\,2\,3\,4)} = \{\sigma : \text{type}(\sigma) = (4, 1)\}$
        \item $\overline{(1\,2\,3\,4\,5)} = \{\sigma : \text{type}(\sigma) = (5)\}$
    \end{enumerate}
\end{enumerate}

\section*{Problem 3}

\textbf{Classification of Groups of Order at Most 10:}

Using the fundamental theorem of finite abelian groups:
\begin{itemize}
    \item For $n < 10$ prime: $G$ is cyclic, $G \cong \mathbb{Z}/n\mathbb{Z}$ for $n \in \{2, 3, 5, 7\}$.
    \item $n = 4$: If $G$ is abelian, $G \cong \mathbb{Z}/4\mathbb{Z}$ or $V_4$ (Klein four-group).
    \item $n = 6$: If $G$ is abelian, $G \cong \mathbb{Z}/6\mathbb{Z}$ or $\mathbb{Z}/2\mathbb{Z} \times \mathbb{Z}/3\mathbb{Z}$. If $G$ is non-abelian, $G \cong S_3$.
    \item $n = 8$: If $G$ is abelian, $G \cong \mathbb{Z}/8\mathbb{Z}$ or $\mathbb{Z}/2\mathbb{Z} \times \mathbb{Z}/4\mathbb{Z}$ or $\mathbb{Z}/2\mathbb{Z} \times \mathbb{Z}/2\mathbb{Z} \times \mathbb{Z}/2\mathbb{Z}$.

         \item If $n = 9$ and $G$ is abelian, $G \cong  \Z{9} $ or $G \cong \Z{3} \times \Z{3}$. If $G$ is non-abelian, then $G$ is not cyclic and there force contains $3$ cyclic sub-groups. Since $3$ is prime, two sub-groups have the trivial sub-groups as intersection. WLOG, let $\langle a\rangle, \langle b\rangle, \langle c\rangle, \langle d\rangle$ be the three subgroups of $G$. Then $ab$ can not be $e, a^{\pm 1}, b^{\pm 1}$. WLOG, let $ab = c$. This implies $ab = c$, $ac = d$ and $ad = b$. If $ab = ba$, then $ac = a^2 b = aba = ca$ and $ad = a^2ca = aca = da$, which implies $a \in Z(G)$ making $G$ commutative. Thus $ba = d$. But then $ad = b = a^2d = a^2ba = aca = da$ and by similar argument as above, $G$ becomes abelian. Thus there is no non-abelian group of order $9$.

        
         \item Let $n = 8$. If $G$ is abelian, then $ G $ is isomorphic to either $\Z{8}$, $\Z{4} \times \Z{2}$ or $\Z{2} \times \Z{2} \times \Z{2}$. If $G$ is non-abelian, by Lagrange's theorem, then there is at least on element $a$ of order $4$. If $\langle a \rangle$ is cycle of order $4$, and if all elements $b \in G - \langle a \rangle$ have order $2$, then $G \cong D_4$. Let $G$ contain two elements $a, b$ of order $4$. If $\brk{a} \cap \brk{b} = \{1\}$, then $ab = ba $ is an element of order $2$. But $e = abab = aabb = a^2b^2 \implies a^2 = b^2$. Thus $\brk{a} \cap \brk{b} = \{1, a^2\}$ since $a^{\pm 1}$ is a generator of $\brk{a}$. We know $ab, ba \in G - \brk{a} - \brk{b}$. Since $a^2 = b^2$, $aabb = e$ or $ab = a^3b^3 = (ba)^{-1}$. Thus $G\cong Q_8$. 
    \end{itemize}

\section*{Problem 4}
\begin{enumerate}[(i)]
    \item State the second isomorphism theorem.

    Let $G$ be a group and $H \leq G$ and $K \unlhd G$. Then, 

    $$\dfrac{HK}{K} \cong \dfrac{H}{H \cap K}$$

    \item Prove the second isomorphism theorem.

    Define a mapping $\phi: H \to HK/K$, as 
    $$\phi(h) = hK.$$

    Note that $\phi$ is surjective because if $a = hK \in HK/K$, then $\phi(h) = a.$ If $a, b \in H$, then $\phi(ab) = abK = aK\cdot bK = \phi(a)\phi(b)$. Thus $\phi$ is a homomorphism. Now, 
    $\ker \phi = \{h \in H: hK = K\} = \{h \in H: h \in K\} = H \cap K$. By the first Isomorphism theorem, the theorem follows.
    \end{enumerate}

    \section*{Problem 5}

    \begin{enumerate}[(i)]
        \item State Sylows theorems. 

        \begin{enumerate}
            \item \textit{First Sylow Theorem}: Let $n$ be the order of a group $G$ and $n = p^am$ where $p$ does not divide $m$. Then there is a subgroup of $G$ with order $p^a$ known as a Sylow $p$-group.
            \item \textit{Second Sylow Theorem}: If $P$ and $Q$ are two Sylow $p$-subgroups of $G$, then $P$ and $Q$ are conjugates of each other.

            \item \textit{Third Sylow Theorem}: If $n_p$ is the number of Sylow $p$-subgroups of $G$, then $n_p \equiv 1 \ \mathrm{mod}\  p.$
        \end{enumerate}

        \item Let $G$ be a group of order $pqr$, where $p$, $q$ and $r$ are distinct primes. Show that $G$ is not simple.

        WLOG, let $p < q < r$. By the third Sylow theorem, $n_r = 1 + kr$ for some $k > 0$ by assumption. Since $n_r > r, p, q$, $n_r | pqr$ and $n_r \nmid r$, $n_r = pq$. However, this means there are $pq(r-1)$ distinct elements in the Sylow $r-$group of $G$. Let $n_p = 1 + p$, and $n_q = 1 + q$, yielding $(p - 1)(p + 1) = p^2 - 1$ and $(q-1)(q+1) = (q^2 - 1)$ distinct elements (resp.) in the Sylow $p$-subgroups and Sylow $q$-subgroups (resp.). Thus we have

        $$pqr \geq pqr - pq + p^2 + q^2 -1.$$
        Clearly, $q^2 > pq  - p^2 = p(q - p) + 2$, yielding a contradiction and thus one of $n_p, n_q$ and $n_r$ is $1$.
    \end{enumerate}


\section*{Problem 6}
\begin{enumerate}[(i)]
    \item If the prime ideal $P$ contains the product $IJ$ of two ideals then prove that $P$ contains either $I$ or $J$.
    
    If $J \subseteq P$, there is nothing to prove, so let $J \nsubseteq P$. There is an element $j \in J$ not in $P$. We know $\{ij : i \in I\} \subseteq IJ \subseteq P$. Since $P$ is prime, this means $I \subseteq P.$ 
    \item Exhibit a natural bijection between the prime ideals of $R/IJ$ and $R/I \cap J$.

    Let $S_G$ be the collection of prime ideals in the group $G$. I claim that the mapping $\pi: S_{R/IJ} \to S_{R/I\cap J}$ defined by
    $$\pi(P/IJ) = P/I\cap J$$
    is a bijection. By (i), if $IJ \subseteq P \in S_{R}$, then $I \cap J \subseteq P.$ Therefore, $\pi$ is defined on the whole of $S_{R/IJ}.$ It remains to prove if $P$ is a prime ideal that contains $I \cap J$, then it should also contain $IJ$. This follows from noting that if $i \in I$ and $j \in J$, then $ij \in I \cap J$ and all finite sums of the form $\sum ij \in I \cap J$, implying $IJ \subseteq I \cap J$.
    \item Give an example of a ring $R$, and ideals $I$ and $J$ such that $IJ$ and $I \cap J$ are different.

    Consider the ideals $I =  \brk{2}$ and $J = \brk{4}$ in the ring $R = \mathbb{Z}.$ $IJ \subseteq \brk{8} \subset \brk{4} = I \cap J.$
\end{enumerate}

\section*{Problem 7}

Does every UFD $R$, which is not a field, contain infinitely many irreducible elements which are pairwise not associates? If your answer is yes then prove it and if no then give an example.\\ 


No. DVTs are UFDs with finitely many primes(irreducibles) but infinite elements. 

\section*{Problem 8}

Give an example of an integral domain such that every element of $R$ can be factored into irreducibles and yet $R$ is not a UFD.
\\

\noindent $R = \mathbb{Z}[\sqrt{-5}]$ is an integral domain that is not a UFD. To show that $6$ can be written as a product of different pairs of irreducibles, as $6 = 2\cdot3$ and $6 = (1 + \sqrt{-5})(1 - \sqrt{-5})$. We prove that $2, 3, 1 \pm \sqrt{-5}$ are irreducibles in $R$. To do that, we define a function(norm) $N: R \mapsto \mathbb{Z}$ as $N(a + b\sqrt{-5}) = a^2 + 5b^2.$ It is clear that $N(zw) = N(z)N(w).$ $N(2) = 4= 2\cdot 2$. Since $N(a +  b\sqrt{-5}) = a^2 + 5b^2 \neq 2$ for $a, b \in \mathbb{Z}$, $2$ must be irreducible. Similarly, $3$ and $1 \pm \sqrt{-5}$ are irreducible in $R$. Thus $R$ is not a UFD. 


\section*{Problem 9}
\begin{enumerate} [(i)]
    \item  Show that $\mathbb{Z}[i]$ is a Euclidean domain. \\ 

    \noindent Define an evaluation norm $v:\mathbb{Z}[i] \mapsto \mathbb{N} \cup \{0\}$ as $N(a + bi) = a^2 + b^2.$ Clearly, $v (0) = 0$ and $v(zw) = zw\overline{zw} = z\overline{z}w\overline{w} = v(z)v(w)$. Now we prove that if  $0 \neq z, w \in \mathbb{Z}[i]$, then $z = qw + r$ such that $v(w) > v(r) \geq 0.$  Let $x + yi = z/w$ and let $p + si$ be an element of $\mathbb{Z}[i]$ such that $|x - p| \leq 1/2$ and  $|y - s| \leq 1/2$. If $r = z - w(p + si)$, then
    \begin{eqnarray*}
        v(r) & = & v(z - w(p + qi)) \\
         & = & z - w(p + qi)\overline{(z - w(p + qi))}\\
         & = & w(z/w - (p + qi))\overline{w(z/w - (p + qi)}\\
         & = & w\overline{w}(z/w - (p + qi)\overline{(z/w - (p + qi)}\\
         & \leq &v(w)(1/4  + 1/4)\\
         & < & v(w).
    \end{eqnarray*}

    This proves  $\mathbb{Z}[i]$ is a Euclidean domain.

    \item  Is $6 - i$ prime in $\mathbb{Z}[i]$?

    Since $v(zw) = v(z)v(w)$, if $z \mid 6 - i$, $v(z) \mid v(6 - i) = 37$. But $37$ is a prime number and it's only divisors are $1$ and itself, making $6 - i$ irreducible. Since $\mathbb{Z}[i]$ is Euclidean domain, and therefore PID, $6 - i$ is then prime.

    \newcommand{\bb}[2]{\mathbb{#1}_{#2}}
    \section*{Problem 10}
    Write down all irreducible polynomials of degree 2 over the field $\bb{F}{5}$.\\

    \noindent Let $p(x) = ax^2 + bx + c, a\neq 0, b, c \in \bb{F}{5}.$ We consider two cases: \textit{Case 1}: $b = 0$. Substituting, $x = \pm 1$ and $x = \pm2$ in $p(x)$, we obtain $a \neq \pm c$. Hence, the polynomials $x^2 \pm 2$ and their associates are irreducible. \textit{Case 2}: if $b \neq 0$, we obtain $a + b + c \neq 0, -b, \pm 2b$, leaving only $a + b + c = b \implies a = -c$. In this case, the polynomial we need to investigate has the form $ax^2 + bx - a = a(x^2 + a^{-1}bx - 1)$. This implies $p(x)$ is irreducible iff $x^2 + b'x - 1$ is irreducible for $b' \neq 0.$ Considering the latter, case by case, we obtain $x^2 \pm2x -1 $ is irreducible. Hence The irreducible polynomials (upto multiplication by units) over the field $\bb{F}{5}$ are $\{ x^2 \pm  2, x^2 \pm 2x - 1\}$

\end{enumerate}

\section*{Problem 11}
\begin{enumerate}[(i)]
    \item State Gauss’ Lemma and Eisenstein’s criteria.\\
    
    \begin{itemize}
        \item \textit{Gauss' Lemma}: Let $D$ be an integral domain and let $F$ be its field of fraction. If $p(x)$ is a monic polynomial in $D[x]$ and it can be factorized in to two polynomials $f(x), g(x)$ in $F[x]$ as $p(x) = f(x)g(x)$, then $f(x)= u(x)v(x)$ such that $u(x), v(x) \in D[x]$ and $\deg f(x) = \deg u(x)$ and $\deg g(x) = \deg v(x).$

        \item \textit{Eisenstein's Criterion}: Let $p(x) = \sum_i a_ix^i \in \mathbb{Z}[x]$ and let $p$ be a prime. If $p \mid a_i$ for $0 \leq i < n$, $p \nmid a_n$ and $p^2 \nmid a_0$, then $p(x)$ is irreducible in $\mathbb{Z}[x]$ and hence in $\mathbb{Q}[x].$
    \end{itemize}
    \item Show that the polynomial $1 + x^3 + x^6 \in \mathbb{Q}[x]$ is irreducible (Hint: try a substitution.)\\

    \noindent Let $p(x) = 1 + x + x^2.$ The given polynomial is irreducible iff $p(x)$ is irreducible in $\mathbb{Q}[x].$ But $p(x)$ is irreducible if $p(x+1) = x^2 + 3x + 3$ is irreducible which it is by Eisenstein's criterion stated in (i) by taking $p = 3.$

    \item Show that the polynomial $1 - t^2 + t^5$ is irreducible over $\mathbb{Q}$ (Hint: consider the ring $\mathbb{F}_{2}[t]$.)\\

    \noindent First, we observe that if a monic polynomial is not irreducible in $\mathbb{Q}[x]$, then it is clearly not irreducible in $\mathbb{F}_p(x)$. To show that let $p(x)$ factorize into $p_1(x) \cdots p_n(x)$ in $\mathbb{Z}[x]$. Then $p(x) + \brk{p} = \big(\prod_i p_i(x) \big) + \brk{p} = \prod_i (p_i(x) + \brk{p}) \in \mathbb{F}_p(x)$. Hence, by contra-positive, $p(x)$ is irreducible in $\mathbb{Q}[x]$ if it is irreducible in $\mathbb{F}_p[x]$. Taking the special case, $p = 2$, $f(x) = 1 - x^2 + x^5 \neq 0$ for $x \in \{\pm 1, 0\}.$ Hence, $f(t)$ must be irreducible in $\mathbb{Q}[t].$
    

\end{enumerate}



\end{document}
